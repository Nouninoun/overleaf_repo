\begin{conclusion}

%le but => Faciliter la modélisation des mecha dans le proc de pre conception d'un jv
Un projet de développement informatique est une tâche de longue haleine rythmée par de nombreux changements,
et
la description des concepts et la conception est essentielle au bon déroulement du projet ainsi qu'à sa réussite.
%
Ceci encore plus marqué dans le domaine du développement de jeux vidéos car
le développement d'un jeu implique non seulement des développeurs mais également de nombreux autres corps de métier~: artistes (planches pour illustrer les éléments graphiques),  musiciens (bande originale du jeu), modeleurs et animateurs 3D (éléments graphiques), scénaristes (cadre et histoire du jeu), etc.
Réussir à apporter les informations nécessaires à tous ces corps de métier est un défi que le \emph{game designer} affronte au quotidien.

Un document de \emph{game design} est un pilier d'un projet de développement d'un jeu.
Sa rédaction est complexe, chronophage et nécessite une rigueur exemplaire des \emph{game designers}.
Cependant, aucun gabarit préconçu n'existe pour répondre à tous leurs besoins.
Chaque projet est différent, chaque jeu possède des spécificités, chaque équipe de développement est différente et cela empêche l'établissement d'un modèle générique de \emph{Game Design Document}.

Dans la littérature, nous avons identifié plusieurs listes de bonnes pratiques de design, chacune essayant de généraliser ses conseils pour tous les genres de jeu.
C'est le cas du \emph{framework MDA}, qui propose un découpage d'un jeu afin de structurer les différents éléments dans trois catégories et lier ces éléments pour qu'ils fonctionnent ensemble.

Définir tous les éléments d'un système et décrire leurs interactions est une approche déjà présente dans le développement logiciel classique.
C'est ce que l'on retrouve notamment à travers la modélisation UML.
Cependant, UML est un langage générique, conçu pour le développement informatique et il est donc compliqué pour les autres corps de métier d'acquérir les connaissances nécessaires à sa compréhension.
Il est cependant possible de créer des profils UML afin d'adapter les modèles UML et les étendre pour intégrer un vocabulaire spécifique à un domaine d'activité particulier.

Une fois le vocabulaire adapté à la conception de jeux vidéos, UML a la capacité d'apporter plusieurs avantages.
UML est un langage formel, permettant la description de systèmes et de processus, efficace, extensible et largement documenté.
Les outils qui l'accompagnent sont fiables et respectent une rigueur qui permet la réutilisation des modèles et en respectant le contenu des données.
Sa formalisation permet également d'assurer la pérennité des modèles dans le temps et leur cohérence, peu importe le nombre de modifications et la durée de leur utilisation.

\gt{Il faut mettre un tilde <<\~{}>> devant $\backslash$ref et
$\backslash$cite, mais pas $\backslash$emph!}


\gt{Ci-bas: encore <<pré>>-conception, qui n'apparaît pas dans les étapes du chapitre I!}

C'est ce que nous recherchions dans notre problématique : apporter une description fiable, modifiable, compréhensible et précise dans la conception d'un jeu vidéo et la rédaction de son \emph{Game Design Document}.
%
Et c'est dans cette optique que nous avons conçu \emph{Game Genesis}, un profil UML intégrant un vocabulaire adapté à la conception de jeux vidéos dans des modèles UML.
%
Pour ce faire, 
nous avons étudié différents \emph{Game Design Documents} afin de définir des caractéristiques communes à ces documents.
Nous nous sommes concentrés sur la modélisation des éléments de \emph{Mechanics} présents dans ces documents afin d'établir une liste des éléments qui se retrouvent dans de nombreux genres et donc de nombreux projets de développement de jeux vidéos.
Une fois cette liste établie, nous avons proposé une liste de stéréotypes correspondant à ces éléments, 
liste dont les éléments ont été organisés selon une structure hiérarchique afin de les classifier dans des catégories conceptuelles appropriées.
Afin de représenter, cette hiérarchie nous avons utilisé le mécanisme d'héritage d'UML.
Ceci nous a permis de définir des caractéristiques dans les catégories et les éléments qui leur sont associés héritent des attributs.

Afin de décrire les interactions entre les éléments de \emph{Mechanics}, nous avons utilisé des associations.
Les stéréotypes de \emph{Game Genesis} comprennent donc une catégorie \texttt{Interaction} dont les éléments s'appliquent sur les associations UML.
Ces stéréotypes décrivent les interactions entre les éléments et peuvent porter des attributs et méthodes.

Finalement, nous avons exploré l'utilisation du profil \emph{Game Genesis} dans un exemple à travers le jeu \emph{PlayerUnknown's BattleGrounds} (PUBG).
Nous avons modélisé certains des concepts clés d'une partie de ce jeu en utilisant un diagramme de classes sur lequel était appliqué le profil \emph{Game Genesis}.
Donc, plus spécifiquement, avec ce modèle, nous avons décrit une partie des éléments de \emph{Mechanics} du jeu PUBG et certaines de leurs interactions.

\emph{Game Genesis} ne se concentre que sur la description des éléments de \emph{Mechanics} d'un jeu.
Or, ce n'est qu'une partie du design d'un jeu, de la rédaction d'un \emph{Game Design Document} et du \emph{Frameowrk MDA}.
Nous nous sommes concentrés sur ces éléments de façon à limiter l'envergure du travail, car la représentation de tous les éléments du \emph{framework MDA} aurait été trop complexe à réaliser dans le cadre d'un travail de maîtrise.
%
On peut toutefois envisager que d'autres types de diagrammes UML
puissent être utilisés pour le \emph{design} de jeux, plus précisément
pour la description des éléments de \emph{Dynamics}.

La représentation des éléments et concepts d'un jeu dans un diagramme de classes ouvre de nombreuses possibilités pour la rédaction de \emph{Game Design Documents}.
En effet, il est possible de réutiliser les modèles, et UML étant un langage informatique, il est possible de le manipuler et de générer d'autres documents.
En réutilisant les modèles, il serait possible d'extraire des informations des modèles afin de générer le début d'un \emph{Game Design Document} décrivant les éléments de \emph{Mechanics} du jeu en question.
Il serait également possible de générer un squelette d'application à partir des modèles, un diagramme de classe étant tout indiqué pour cette transformation.
Nous pourrions imaginer un outil de gestion des versions permettant de faire un différentiel entre les différentes versions des modèles.
Il serait alors possible de stocker les justifications d'une modification effectuée et documenter ces modifications pour le futur.
%
Par contre, toutes ces possibilités sont pour l'instant des pistes car \emph{Game Genesis} est aujourd'hui à l'état de preuve de concept.
Ces possibles évolutions demandent que notre profil
\emph{Game Genesis} soit incorporé de façon effective dans un outil
UML existant, ce que nous n'avons pas eu le temps de faire, et qui
serait une piste intéressante pour un travail futur.


\gt{Ci-haut: il faut aussi dire quelques mots sur une autre limitation
--- il vaut mieux le dire explicitement, plutôt que se le faire dire
par un examinateur: ce que tu as fait est une sorte de prototype ou
preuve de concept exploratoire, en ce sens que ce profil n'a pas été
inclus de façon effective dans un vrai outil UML.  C'est aussi une
piste de travail futur à mentionner.}

\gt{Et ci-bas: il ne faut pas trop dire que c'est vraiment un outil,
car beaucoup plus une preuve de concept pour l'instant. Peut-être plus
des <<pistes intéressantes>>?}


Finalement, nous espérons que cette recherche aura permis d'identifier des pistes intéressantes et nouvelles pour concevoir un outil adapté au domaine du développement de jeux vidéos, avec comme objectif d'apporter une représentation graphique des éléments de \emph{Mechanics} --- et possiblement de \emph{Dynamics} --- et  permettant de structurer et de documenter le processus de conception et d'accompagner les \emph{game designers} dans leur tâche.


%la nature et l’envergure du travail
%les sujets traités => Dev jv, gdd, mda, profils
%les problèmes à résoudre => formaliser une représentation des Mecha, faciliter la representation, 
%les objectifs fixés => apporter un outil fiable, langage adapté aux JV, malléable pour tous les gameplay, 
%les méthodes utilisées => profil UML,Établir un profil UML pour représenter les éléments de Mechanics dans le processus de pré-conception d'un JV
%la démarche adoptée => GDD (exemples et bonnes pratiques), MDA (section Mecha), Profil UML
%les résultats les plus saillants => Exemple d'application ok, possibilité de decrire les mecha et leurs relations efficacement
%les limites et les conclusions => Représentation des Mecha (no dyna no aesth)
%les recommandations et les pistes de recherche => utilisation des modeles générés pour : générer un squelette de GDD, générer un squelette de code, stocker les données pour le versionning


\end{conclusion}


