\chapter{Un profile UML pour aider à la rédaction d'un GDD}

\label{game-genesis.sect}

\textcolor{red}{Trouver une manière d'introduire le chapitre}


\section{}

\textcolor{red}{Trouver une manière d'introduire le chapitre}
Ne modifient pas le metamodele
Un profile étend le metamodel existant via des stereotypes, des tagged values et des contraintes.
Pas possible de modifier les contraintes déjà existantes mais permet l'adaptation et la custromisation d'éléments existants du metamodel
Les customisations sont définies dans un profile qui est ensuite appliqué à un package.
Il est possible d'appliquer les profiles dynamiquement aux modeles et ils peuvent également être combinés pour appliquer plusieurs profiles simultanés au même modèle.
%%%%%%%%%%%%%%%%%%%%%%%%%%%%%%%%%%%%%%%%%%%%%%%%%%%%%%%%%%%%%%%%%%%%%%%%%%
\begin{comment}
\chapter{Un langage de modélisation pour l'établissement d'un Game Design Document}

\section{Le concept}
\subsection{Définition}
\subsubsection{Quoi ?}
Un langage permettant de modéliser et stocker des idées lors des phases de Breakthrough et de Conception d'un projet de jeu vidéo. La modélisation peut être graphique et/ou textuelle avec application des modifications en parallèle. \\
Les informations peuvent contenir tout le nécessaire pour exprimer les idées (textes, informations numériques, chemins de fichiers...). Les champs peuvent être personalisables pour permettre de la souplesse aux utilisateurs.

\subsubsection{Pour quoi faire ?}
\paragraph{Des outils de modélisation existent pour tous les domaines reliés au développement de logiciels. Ils sont souvent spécifiques à un corps de métier afin de pouvoir proposer un maximum de fonctionnalités spécifiques sans devenir trop compliqué et en utilisant un vocabulaire précis qui correspond au corps de métier concerné.}

\paragraph{Il y a peu ou pas de langages de modélisation plus généraux pour des domaines multi-métiers. Le but est de pouvoir modéliser la réflexion créative en fournissant un élément visuel permettant de mind-mapper les idées, les stocker et les réutiliser. \\
Il faut que la modélisation soit assez souple pour pouvoir répondre aux besoins de chacun des corps de métier d'où le fait que les éléments et attributs peuvent avoir des identifiants spécifiques définis librement par l'utilisateur.}

\subsubsection{Pour quelles raisons ?}
\paragraph{Les supports de réflexion actuellement utilisés : cahier des charges, réunions, notes écrites, mails, minds-maps... L'organisation de ces différents supports dans un ensemble cohérent est tr;s compliqué. Dans un cahier des charges il est compliqué de classer les idées à la volée. Un mind-map nécessite une numérisation ou une retranscription sur un outil de mind-mapping qui sont toutes les deux des techniques non péreines et risquées dans la conservation des données. Des notes écrites peuvent se perdre et n'ont pas de durabilité sur le long terme. Des mails sont péreins mais il est difficile de les organiser pour le stockage de l'information.}
\paragraph{Un langage de modélisation graphique et textuel permettrait de mind-mapper les idées à la volée sous forme de cubes contenant les données nécessaires. La hiérarchisation des éléments permettrait de gérer des héritages et des relations ainsi que d'éviter la répétition trop abondante des mêmes informations. Les faces des cubes permettrait d'isoler les informations nécessaires à chacun des corps de métier.}
\end{comment}
