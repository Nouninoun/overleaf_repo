\chapter{Les profiles UML}
\textcolor{red}{Trouver une manière d'introduire le chapitre}
Ne modifient pas le metamodele
Un profile étend le metamodel existant via des stereotypes, des tagged values et des contraintes.
Pas possible de modifier les contraintes déjà existantes mais permet l'adaptation et la custromisation d'éléments existants du metamodel
Les customisations sont définies dans un profile qui est ensuite appliqué à un package.
Il est possible d'appliquer les profiles dynamiquement aux modeles et ils peuvent également être combinés pour appliquer plusieurs profiles simultanés au même modèle.