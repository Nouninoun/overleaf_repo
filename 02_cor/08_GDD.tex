\chapter{Les documents de game design}

Les documents de game design sont nombreux et très discutés. Selon le niveau d'avancement du développement certains sont plus présents que d'autres mais le domaine est d'accord pour avancer qu'il n'existe pas de template fixe pour un document de game design, encore plus, qu'un template trop précis serait une erreur à ne pas commettre car il entraînerait des difficultés de rédaction et d'adaptation à tous les types de jeux que l'on peut rencontrer. La littérature s'accorde pour dire qu'il n'y a pas de template définit pour les documents de game design mais certaines lignes directrices peuvent être suivies afin que le game design document soit complet et respecte une certaines forme.\\
Dans sa présentation Librande montre des exemples de livrables de game design de différentes sortes. Il souligne que les documents de design lourds sont des sources d'informations importantes, réunissant tout le design dans un seul document et la création du document aide grandement à designer le jeu par la suite. Cependant ces documents sont compliqués à maintenir, mettre à jours et à parcourir pour trouver une information précise.\\
Il présente également les "design wiki" qui apportent également beaucoup de points positifs au design. Il est possible d'y avoir accès à n'importe quel endroit tant qu'une connexion est disponible. Les mises à jours sont simplifiées car la recherche l'est également : il devient alors possible de modifier des éléments en "live" lors d'une réunion. Un Wiki permet la contribution d'équipe aux différents éléments, il est possible d'éditer les éléments par article et donc de ne pas avoir à prendre connaissance d'éléments non reliés au travail actuellement effectué. Il est facile de maintenir un versionning et un historique des modifications pour justifier les modifications et les garder en mémoire. Cependant un Wiki requiert une maintenance constante et donc souvent une personne dédiée à la maintenance du Wiki, les relations entre les éléments ne sont pas mises en avant et il est compliqué de mettre ensemble différentes sortes de représentation sous la forme de textes/images dans un Wiki, les images doivent être traitées à l'extérieur du Wiki avant d'y être réintégrées, l'impression d'un article Wiki n'est pas toujours adaptée à une lecture rapide des éléments.\\
%Scott Rogers Level Up!
%Game design foundations, R. Pedersen
%Game design theory & practice Second Edition, Rouse Richard
%A Systematic Review of Game Design Methods and Tools, Gomez, Jaccheri, Hauge
%GAMASUTRA Game Design Methods: A 2003 Survey, Bernd Kreimeier 



\section{Le "one page" : le concept document}
%One page designs, Stone Librante (ppt presentation)
%Game design foundations, R. Pedersen
Le One page design est une vue d'ensemble du jeu qui est destinée à être vue aussi bine par l'équipe de développement que par les acteurs décisionnels des studios il doit alors contenir assez d'informations à mettre en avant sur jeu mais tout doit tenir sur une page [LevelUP!].

C'est à partir de ces éléments que Librande estime que plus un document est long moins les utilisateurs s'y réfèrent, sa solution : Le One-Page Design.\\
L'idée est de représenter tous les éléments nécessaires au design sur une seule page et d'y intégrer toutes les informations nécessaires de manière visuelle. Voici les éléments essentiels cités par Librande :
\begin{itemize}
    \item Un titre représentatif du design
    \item Dater tous les éléments afin de garder un historique
    \item Espacer les informations, ne pas créer des blocs d'informations
    \item Une illustration centrale peut focaliser l'attention
    \item Sous cette illustration il est possible d'ajouter une description et des textes explicatifs
    \item Pour des détails supplémentaires ajouter des légendes autour de l'illustration
    \item Les légendes peuvent être des illustrations elles-mêmes accompagnées de notes
    \item Utiliser des barres latérales pour ajouter des checklists, des objectifs principaux ou des informations
\end{itemize}
Attention, la taille des éléments est importante dans un One-Page Design, les tailles indiquent l'importance de l'information, si nécessaire un One-Page Design peut être étendu à la taille nécessaire pour contenir toute l'information à commencer par une taille Legal-US jusqu'aux posters.
Cependant l'augmentation de taille doit continuer à respecter la lisibilité du One-Page Design. Augmenter en taille ne doit pas correspondre à intégrer trop d'informations dans un One-Page car cela irait à l'encontre de toute la lisibilité du design.


\section{Le "five pager" : l'executive summary }
%Game design foundations, R. Pedersen
Le five pager est un résumé du concept du jeu vidéo et une description du jeu à venir. Il comprend toutes les informations essentielles au cycle de vie du jeu sous des aspects majeurs, tels que le gameplay, l'audience cible, le scénario, et les features. De taille idéale il est constitué d'environ 5 pages, et permet de présenter le jeu sous sa forme la plus basique à un décideur ou à un financeur.

\section{Le Game Design Document}
%Game design foundations, R. Pedersen
%Proposal of Game Design Document from Software Engineering Requirements Perspective, Mario Gonzalez Salazar, Hugo A. Mitre, Cuauhtémoc Lemus Olalde, José Luis González Sánchez

%Incorporating a Game Design Document into Game Development Projects Deliverables,Craig Marais,Lynn Futcher,Johan van Niekerk
%GAMASUTRA Creating a great design document Tzvi Freeman

%Improving Game Development Process Applying Multi-View Game Design Documents,Pablo Correa , Luisa Möller 
%6


%GAMASUTRA 5 Alternatives to a Game Design Document, Erin Robinson

Le Game Design Document est un des documents les plus complet trouvable dans le cadre du développement d'un jeu vidéo, il est qualifié par certains de "bible" des jeux vidéos. Il doit contenir, dans l'idéal, toute la vision du game designer sans laisser aucun doute sur ce que celui-ci recherche à travers le jeu. Peut importe le corps de métier lisant le game design document celiui-ci doit être capable de se représenter au maximum le rendu final attendu. Il doit être et rester un élément de communication crucial entre les différentes équipes entourant le développement d'un jeu vidéo et ce sur toutes les étapes depuis la création de l'idée jusqu'au post-mortem.

Cependant il doit permettre toute la liberté nécessaire aux designers de créer de nouveaux jeux, de nouveaux gameplay, de nouveaux concepts sans les brider sous quelque forme que ce soit. Beaucoup de travaux essaient de répondre à ce besoin de formalisation du Game Design Document sans pour autant réussir à apporter une solution permettant d'englober tous les types de jeux, tous les types de métiers ou toutes les étapes de développement. De nombreux problèmes annexes se posent concernant l'universalité d'un modèle de Game Design Document : Est-il complet ? Est-il assez précis ? Est-il assez souple ? Est-il utile à l'équipe ? Effectivement si un document est extrêmement complet, permettant d'intégrer tous les détails voulus par le Game Designer, mais que ce document est lourd à parcourir, difficilement partageable, difficilement maintenable et modifiable sera-t-il vraiment efficace dans son rôle d'aide au développement et à la maintenance d'un jeu vidéo ? (LIBRANDE)

Un Game Design Document est l'atout majeur de la phase de production d'un jeu vidéo et pour qu'il soit utile durant toutes les phases celui-ci doit être impérativement complet et sans ambiguité (Game Development Guidelines: Practices To Avoid Conflicts Between Software and Design). Cependant il doit être assez souple afin de suivre le jeu dans son développement et suivre les changements nombreux et soudains qui peuvent arriver au cours du développement. (creating a great design document). Tvzi Freeman avance même dans son article de Gamasutra que le Game Design Document doit non seulement décrire le corps du jeu vidéo mais son "âme". Construire un Game Design Document selon lui doit être un moyen de représenter le résultat attendu dans les moindres détails pour permettre aux équipes de travailler sur une idée précise en faisant appel à des outils graphiques plutôt qu'à des textes si l'idée peut être expliquée plus précisément et plus rapidement. Mais le Game Design Document doit être un moyen d'expliquer toutes les parties du jeu, le "what", mais également le "how", comment le jeu fonctionne et comment l'interaction joueur-jeu prend vie.