\chapter{Les documents de \emph{Game design}}

\label{chap.design_doc}
\gt{Limiter forme <<passive>>.}

Dans ce chapitre, nous allons pr\'esenter plusieurs documents de \emph{Game design} --- \glsfirst{gdd} --- qui peuvent être présents dans un processus de développement de jeu vidéo.
Ces documents peuvent être rédigés l'un à la suite de l'autre, ou indépendamment l'un de l'autre --- et ils ne sont pas obligatoires dans chaque projet de jeu vidéo.


\gt{Je ne comprends pas <<tr\`es discut\'es>>!?}

Les documents de \emph{Game design} sont nombreux et leurs formes ainsi que leurs contenus sont souvent discut\'es dans la littérature.
Selon le niveau d'avancement du développement, certains documents sont plus utiles que d'autres mais la plupart des acteurs du domaine sont d'accord pour avancer qu'il n'existe pas de gabarit fixe pour un document de \emph{Game design}~\cite{GD_theory_rouse}.
En fait, l'utilisation d'un gabarit trop sp\'ecifique et précis serait une erreur à \'eviter, car cela pourrait entraîner des difficultés de rédaction et d'adaptation face aux divers types de jeux que l'on peut rencontrer.
Cependant, certains auteurs présentent des lignes directrices, un peu comme une recette de cuisine~\cite{LevelUpRogers2014}, qui peuvent être suivies afin que le GDD soit complet et respecte une certaines forme. 

Librande~\cite{onepage_librande} montre des exemples de livrables de \emph{game design} de différentes sortes.
Il souligne que les documents de design lourds sont des sources d'informations importantes, réunissant tout le design dans un seul document et la création du document aide grandement à designer le jeu par la suite.
Cependant, ces documents sont compliqués à maintenir, \`a mettre à jour et à parcourir pour trouver une information précise.


\gt{Qui \c{c}a <<Il>>? Librande?}
\gt{Je ne comprends pas <<la contribution d'\'equipe>>?!}

Librande présente également les <<\emph{design wikis}>> qui apportent de nombreux points positifs au design.
Il est possible d'y avoir accès à n'importe quel endroit tant qu'une connexion est disponible.
Les mises à jours sont simplifiées car la recherche l'est également : il devient alors possible de modifier des éléments directement au cours d'une réunion.
Un \emph{wiki} permet aussi la contribution de tous les membres de l'équipe de développement aux différents éléments car la modification est possible par de nombreux utilisateurs.
Il est possible d'éditer les éléments par article et donc de ne pas avoir à prendre connaissance d'éléments non reliés au travail actuellement effectué.
Il est facile de maintenir un historique des versions et des modifications pour justifier les modifications et les garder en mémoire.
Cependant, un \emph{wiki} requiert une maintenance constante et donc, souvent, une personne dédiée à sa maintenance.
Les relations entre les éléments ne sont pas mises en avant et il est compliqué de mettre ensemble différentes sortes de représentation sous la forme de textes/images.
Autres d\'esavantages~: les images doivent être traitées à l'extérieur du \emph{wiki} avant d'y être réintégrées, et l'impression d'un article \emph{wiki} n'est pas toujours adaptée à une lecture rapide des éléments.\\
%Scott Rogers Level Up!
%Game design foundations, R. Pedersen
%Game design theory & practice Second Edition, Rouse Richard
%A Systematic Review of Game Design Methods and Tools, Gomez, Jaccheri, Hauge
%GAMASUTRA Game Design Methods: A 2003 Survey, Bernd Kreimeier 



\section{Le <<\emph{concept document}>>}

L'id\'ee g\'en\'erale d'un <<\emph{concept document}>> est
d'avoir un document concis décrivant \emph{le concept général d'un
jeu}, mais sans entrer dans les détails, en indiquant uniquement les
principales lignes directrices.


\subsection{Le <<\emph{one-page}>> de Librande}

\gt{<<Le concept document>> ou <<La documentation des concepts du jeu>>?}

%One page designs, Stone Librante (ppt presentation)
%Game design foundations, R. Pedersen
%Scott Rogers Level Up!
Librande~\cite{onepage_librande} estime que plus un document est long, moins les utilisateurs s'y réfèrent.
Afin de répondre à cette problématique, il a propos\'e le \emph{One-Page Design},
un document qui donne une vue d'ensemble du jeu.
Ce document est destiné à l'équipe de développement et aux décideurs des studios de production.
Il doit donc contenir les informations essentielles pour présenter le projet de jeu sous forme visuelle, et ce sur une seule page~\cite{LevelUpRogers2014}.

\gt{Le premier paragraphe et le deuxi\`eme contiennent des \'el\'ements r\'ep\'etitifs, non?}
\eh{J'ai regroupé les deux paragraphes et enlevé la répétition}
\gt{A quoi r\'ef\`ere <<ces \'el\'ements>? Pas clair!?}



\gt{Pas clair que c'est une autre proposition --- Librande vs.\
Rogers!? A clarifier/expliciter!}


\gt{Evite d'utiliser les items de gls directement dans une phrase, car pas clair.  Apr\`es un long tiret ou entre parenth\`ese semble pr\'ef\'erable.}

\gt{Et pas n\'ecessaire d'utiliser partout, car cela alourdit le texte
--- car cela met l'item accentu\'e, ce qui ne me semble pas
appropri\'e. }
\eh{J'ai modifié les *gls pour ne les afficher que la première fois qu'ils apparaissent et je les met tous entre --- aftin d'alléger leur apparition}


Voici les caract\'eristiques essentielles d'un \emph{One-page design} tels que d\'ecrits par Librande~\cite{onepage_librande}:
\begin{itemize}
    \item Un titre représentatif du jeu.
    \item Les dates des divers éléments --- afin de garder un historique.
    \item Des espaces entre les informations --- afin de séparer chaque sujet et de faciliter la compréhension à la lecture.

\gt{Pour <<ne pas cr\'eer des blocs>>? J'aurais plut\^ot tendance \`a
penser que mettre des espaces cr\'ee justement des blocs!}

    \item Une illustration centrale --- pour focaliser l'attention.
    \item Sous l'illustration, possiblement une description et des textes explicatifs.
    \item Des légendes autour de l'illustration --- pour des détails supplémentaires, 
     ces légendes pouvant être des illustrations, elles-mêmes accompagnées de notes.
    \item Des barres latérales --- pour ajouter des \emph{checklists}, des objectifs principaux ou des informations diverses.
\end{itemize}

Selon Librande, la taille des éléments est importante dans un \emph{One-page design}: les tailles indiquent l'importance de l'information.
Si nécessaire, un \emph{One-Page design} peut être étendu à la taille nécessaire pour contenir toute l'information, y compris sous forme d'un poster.
Cependant, l'augmentation de taille de la page doit quand même permettre d'assurer sa lisibilité --- augmenter en taille ne doit pas conduire à intégrer trop d'informations, ce qui irait à l'encontre de la lisibilité.


\subsection{Le <<\emph{one-sheet}>> de Rogers}

\gt{Pas besoin de dire <<Dans son article>>, <<Dans un livre>>,
etc. On le saura via la bibliographie. Tu donnes simplement l'auteur,
comme sujet de la phrase...}


Rogers présente le <<\emph{one-sheet}>> comme vue d'ensemble d'un jeu~\cite{LevelUpRogers2014}.
Il insiste sur le fait que ce document doit être intéressant, informatif et court.
Voici les éléments que doit contenir le \emph{One-Sheet} de Rogers~\cite{LevelUpRogers2014}.
\begin{itemize}
    \item Le titre du jeu.
    \item Les systèmes de jeu prévus.
    \item L'\^age des joueurs visés.
    \item La notation ESRB --- voir plus bas.
    \item Un résumé de l'histoire du jeu en se concentrant sur le \emph{gameplay}.
    \item Les modes du \emph{gameplay}.
    \item Les USP --- voir plus bas.
    \item Les jeux compétiteurs --- voir plus bas.
\end{itemize}

\gt{Jeux et logiciels? Ou juste jeux?}

Le ESRB --- \gls{esrb} --- est un organisme d'autorégulation qui est à l'origine d'un système de notation et de règles de respect de la vie privée des jeux et logiciels.
Son système de notation permet de définir à partir de points précis l'âge conseillé pour l'utilisation d'un jeu ou logiciel.
C'est une notation que l'on retrouve systématiquement dans le domaine du jeu vidéo définissant le type de contenu présent dans le jeu et le public pour lequel est recommandé le contenu : eC (\emph{Early Childhood}), E (\emph{Everyone}), E10 (\emph{Everyone~$10^+$}), T (\emph{Teen}), M (\emph{Mature} $17^+$), AO (\emph{Adults Only} $18^+$).


\gt{Est-ce vraiment des <<points>> au sens de <<pointage>>
(num\'eriques)? Ou juste des <<arguments>> --- autre sens du terme
<<point>> en anglais~: selling point = argument de vente!?}

Les USP --- \gls{usp} --- sont des arguments de vente représentant le jeu.
Ils sont g\'en\'eralement indiqu\'es à l'arrière de la boite de jeu ou sur le descriptif du jeu dans le cas d'une vente dématérialisée.
Ce sont des points précis et courts attirant la curiosité de l'acheteur sans être trop descriptifs. 


\gt{Je ne comprends pas le d\'ebut de la phrase suivante --- produits comp\'etitifs? Comp\'etiteurs?}

Les jeux compétiteurs liste les concurrents actuels du jeu, c'est-\`a-dire, les jeux déjà présents sur le marché.
Cette liste doit contenir des jeux connus ou connaissant un grand succès, afin qu'ils soient représentatifs du type du jeu d\'ecrit par le \emph{one-sheet}.
Cela permet alors de donner une bonne idée de ce que sera le jeu.
Présenter une liste de jeux obscurs ou qui n'ont connu que peu de succès découragera les éditeurs qui liront le \emph{one-sheet}.

\gt{Ci-bas: Comme c'est une liste d'\'elements, il faut que tous
soient des substantifs --- et non des actions (verbes).}



\section{Des extensions du document d'une page vers un résumé}
Une fois les premières étapes de \emph{Game design} effectuées, le document \emph{One-page} ou \emph{One-sheet} va être étendu afin de produire un document plus important.
Ce second document va préciser les concepts du jeu en cours de création et va permettre de communiquer un niveau de détails plus précis.

%Game design foundations, R. Pedersen
Pedersen propose le \emph{five pager}~\cite{GD_foundations_pedersen}. C'est un résumé du concept du jeu vidéo et une description du jeu à venir.
Il comprend toutes les informations essentielles au cycle de vie du jeu sous ses aspects majeurs, tels que le \emph{gameplay}, l'audience cible, le scénario, et les \emph{features}.
Le \emph{five-pager} permet de présenter le jeu de manière plus poussée que le \emph{One-page} à un décideur ou à un éditeur.

\gt{Ci-haut: cela me semble contradictoire/ambigue/m\'elangeant de
dire que le document de 5 pages pr\'esente le jeu sous sa forme <<la
plus basique>>, alors qu'on a au pr\'elable produit un document d'une
seule page. Reformuler?}


\gt{Il faut \'eviter de mettre dans le corps du texte des longues
s\'eries de d\'etails tel que le contenu du \emph{Ten-pager}.  Dans le
cas pr\'esent, je crois qu'une table serait plus appropri\'ee.}


\gt{Pour une telle liste/table, pr\'ef\'erable de ne pas mettre
d'article, donc <<Titre du jeu>> plut\^ot que <<Le titre du jeu>>.  De
plus, comme \'evoqu\'e pr\'ec\'edemment, il faut \^etre uniforme,
i.e., substantifs partout --- et non pas m\'elanger substantifs et
verbes (actions).}


Quant \`a Rogers~\cite{LevelUpRogers2014}, il  propose un document plus \emph{fourni} avec son \emph{Ten-pager}.
Le contenu du document doit non seulement contenir des informations pour l'équipe de développement du jeu, mais également les informations nécessaires pour int\'eresser les éditeurs impliqués dans le projet.
Rogers présente le contenu du \emph{Ten-pager} sous la forme indiqu\'ee dans le tableau~\ref{ten-pager.table}.

\gt{Ci-bas. Il faut utiliser des temps de verbes uniformes. Et j'ai
mis des bullets: plus clair au niveau structure.}

\begin{itemize}

\item La première page contient ce que le \emph{One-Page} décrivait auparavant, soit un aperçu concis des caractéristiques du jeu ainsi qu'un calendrier prévisionnel de d\'eveloppement.

\item Les pages 2 à 8 contiennent les informations sur le jeu en lui même : l'histoire, les personnages, le \emph{gameplay}, l'expérience que le joueur va rencontrer.

\item La page 9 contient ce qui accompagne le jeu mais de l'extérieur : des succès à remplir, \footnote{Selon la plateforme sur laquelle un jeu est publié, il peut être accompagné de <<~succès~>>, qui sont des tâches à accomplir et qui donnent droit à un trophée ou \`a une récompense quelconque.} des secrets et des \emph{easter-eggs}.%
%
\footnote{Un secret peut être un niveau ou un objet collectionnable, alors qu'un \emph{easter-egg} est une blague glissée dans un jeu par un développeur ou par l'équipe de développement.}

\gt{Ci-haut: je ne comprends pas <<des succ\`es \`a remplir>>!ù Reformuler!}

\item La page 10 est un regroupement d'informations pratiques pour les décideurs du projet ou les potentiels financeurs afin d'expliciter les moyens de monétiser le jeu et d'en faire un projet lucratif.

\end{itemize}
    
\gt{Paragraphe ci-haut~:  Tu parles de deux parties avec destinataires distincts. Ce serait bien d'expliciter}

\gt{Ne pas mettre [H] car cela semble laisser un grand bout blanc. On
met H typiquement seulement si la figure/table est petite et qu'on
veut vraiment qu'elle apparaisse au point pr\'ecis sp\'ecifi\'e.}

\begin{table}
\footnotesize
\begin{framed}
\begin{itemize}
    \item Page 1~: Informations générales
    \begin{itemize}
        \item Titre du jeu
        \item Systèmes de jeu prévus
        \item \^Age des joueurs visés
        \item Notation ESRB
        \item Calendrier prévisionnel de sortie
    \end{itemize}
    \item Page 2: Histoire
    \begin{itemize}
        \item  Résumé de l'histoire du jeu permettant de poser les premiers jalons de manière succinte et générale
        \item Résumé du déroulement du jeu  permettant de situer les actions du joueur dans l'histoire, les d\'efis rencontrés par celui-ci, comment se déroule la progression, la place du \emph{gameplay} dans l'histoire, les conditions de victoire, etc.
    \end{itemize}
    \item Page 3 : Détails du personnage contr\^ole par le joueur
    \begin{itemize}
        \item Histoire du personnage --- caractère, traits importants de son apparence
        \item \emph{Gameplay} particulier associé au personnage, ses mouvements, ses armes
        \item Maquette des contrôles proposés au joueur, par ex., représentation des raccourcis claviers
    \end{itemize}
    \item Page 4~: \emph{Gameplay}
    \begin{itemize}
        \item Modèle de séparation de l'histoire (niveaux, chapitres, monde ouvert)
        \item Scénarios particuliers (cinématique active)
        \item Mise en avant des USP
        \item Diagrammes et illustrations apportant des précisions
    \end{itemize}
    \item Page 5
    \begin{itemize}
        \item Images et description du monde du jeu
        \item Découpage des zones de jeu
        \item Liens entre les zones
    \end{itemize}

\end{itemize}
\end{framed}
\caption{Contenu du \emph{Ten-pager} selon Rogers~\cite{LevelUpRogers2014}.}
\end{table}

\addtocounter{table}{-1}

\begin{table}[H]
\footnotesize
\begin{framed}
\begin{itemize}
    \item Page 6~: Expérience de jeu
    \begin{itemize}
        \item Description des émotions et sensations que doit générer l'expérience de jeu
        \item Description de l'interface de jeu et de la manière d'en faire usage

\gt{A quoi r\'ef\`ere <<les>> dans <<les parcourir>>? Pas clair!}

    \end{itemize}
    \item Page 7~:  Mécaniques de \emph{gameplay}
    \begin{itemize}
        \item Mécaniques : Eléments avec lequel un joueur interagit pour effectuer des actions (par ex., levier pour ouvrir une porte)
        \item Dangers : Eléments du monde pouvant tuer ou blesser le joueur mais sans IA (par ex., bloc de pierre qui tombe)
        \item \emph{Power-up} : Objets que le joueur récupère et utilise pour obtenir un avantage (par ex., champignon dans Mario)
        \item Objets de collections : Objets que le joueur collecte mais qui n'influence par directement le \emph{gameplay} (par ex.,  monnaie)
    \end{itemize}
    \item Page 8~: Ennemis
    \begin{itemize}
        \item Description des ennemis rencontrés par le joueur et qui sont contrôlés par une IA
        \item Description des \emph{boss} rencontrés
    \end{itemize}
    \item Page 9~: Multijoueur et bonus
    \begin{itemize}
        \item Description des succès collectionnables
        \item Description des secrets découvrables
        \item Description des interactions si le jeu est multijoueur
    \end{itemize}
    \item Page 10: Monétisation
    \begin{itemize}
        \item Description du système de monétisation du jeu (Gratuit, Gratuit avec une boutique en jeu, payant à l'achat, etc.)
        \item Description des boutiques et de leur contenu
    \end{itemize}
\end{itemize}
\end{framed}
\caption{Contenu du \emph{Ten-pager} selon Rogers~\cite{LevelUpRogers2014} (suite).}
\label{ten-pager.table}
\end{table}




\section{Le \emph{Game Design Document}}
\label{sect.GDD}
%Game design foundations, R. Pedersen
%Proposal of Game Design Document from Software Engineering Requirements Perspective, Mario Gonzalez Salazar, Hugo A. Mitre, Cuauhtémoc Lemus Olalde, José Luis González Sánchez

%Incorporating a Game Design Document into Game Development Projects Deliverables,Craig Marais,Lynn Futcher,Johan van Niekerk
%GAMASUTRA Creating a great design document Tzvi Freeman

%Improving Game Development Process Applying Multi-View Game Design Documents,Pablo Correa , Luisa Möller 
%6


%GAMASUTRA 5 Alternatives to a Game Design Document, Erin Robinson

\subsection{Le GDD, un moyen de communication}

\gt{L'utilisation de glsfirst semble permettre, quand on trouve cela
appropri\'e, de mettre \`a nouveau de fa\c{c}on explicite le texte
descriptif de l'acronyme. Ici, cela me semblait indiqu\'e.}

Le \glsfirst{gdd} est un des documents les plus complets utilis\'es dans le cadre du développement d'un jeu vidéo.
Dans la littérature, il est souvent qualifié de <<Bible du design>>~\cite{GD_foundations_pedersen} --- bien que certains auteurs ne soient pas d'accord sur ce sujet~\cite{LevelUpRogers2014}. 

Le GDD doit contenir, dans l'idéal, toute la vision du \emph{game designer} sans laisser de doute sur ce que celui-ci veut représenter et souhaite voir dans le jeu.
Peu importe le corps de métier de la personne lisant le GDD, celle-ci doit pouvoir se représenter au maximum le rendu final attendu.
Le GDD doit donc pouvoir servir de moyen communication entre les différentes équipes entourant le développement du jeu, et ce \`a toutes les étapes, depuis la création de l'idée jusqu'au post-mortem du projet.

\subsection{Une structure souple mais un contenu précis}

\gt{Lorsqu'on indique un auteur, il suffit de mettre le nom de famille, sans le pr\'enom.}

Freeman, dans un article sur Gamasutra~\cite{gama_greateGDD}, établit une liste de 10 pratiques qu'il estime nécessaire de respecter pour d\'ecrire un GDD :
\begin{itemize}
    \item Décrire le corps du jeu, mais également son âme.
    \item Rendre le GDD lisible.
    \item Etablir des priorité dans les tâches.
    \item Entrer au maximum dans les détails.
    \item Décrire précisément les idées compliquées des concepts, quitte à les démontrer, les illustrer ou en faire des mini prototypes \gt{Qu'est-ce qu'un <<sujet>>!?}
    \item Décrire le <<quoi>> mais également le <<comment>>.
    \item Proposer des alternatives de mise en \oe{}uvre. \gt{Pas clair: des alternatives de mises en oeuvres?}
    \item Donner une vie au GDD.
    \item Pr\'eciser la vision du \emph{Game designer} de fa\c{c}on suffisamment précise pour ne pas laisser d'éléments non décrits.
    \item Mettre à disposition le GDD dans de bonnes conditions.
\end{itemize}

\subsection{Un GDD complet mais pas surchargé}

\gt{Ci-bas~: j'ai reformul\'e la premi\`ere phrase car c'\'etait vague
de parler <<d'outils>> et c'\'etait bizarre de m\'elanger outils et
libert\'e: A revoir!}

Le GDD doit \^etre un outil pour les \emph{game designers}, tout en leur donnant la liberté de créer de nouveaux jeux, de nouveaux \emph{gameplay}, de nouveaux concepts, donc sans les brider dans leur cr\'eativit\'e.
De nombreux travaux essaient de répondre au besoin de formalisation du GDD~\cite{GDD_software,multiview,GDD_GDProject,gama_greateGDD} sans pour autant réussir à apporter une solution permettant d'englober tous les types de jeux, tous les types de métiers ou toutes les étapes de développement.
Et de nombreux problèmes annexes se posent concernant l'universalité d'un modèle de GDD : Est-il complet ? Est-il assez précis ? Est-il assez souple ? Est-il utile à l'équipe ? 

Prenons l'exemple d'un projet où le GDD est extrêmement complet, comportant tous les détails voulus par le \emph{Game designer}.
Le GDD peut alors devenir tellement complet qu'il en devient imposant, lourd à parcourir, difficile à maintenir et à modifier.
Il perdra alors toute l'efficacité de design qu'il aurait pu donner, pouvant devenir un handicap plutôt qu'une aide au développement~\cite{onepage_librande}.

Un GDD est l'atout majeur de la phase de production d'un jeu vidéo et pour qu'il soit utile durant toutes les phases, celui-ci doit être impérativement complet et sans ambiguité~\cite{GD_Guidelines}.
Cependant, il doit être assez souple afin de suivre le jeu dans son développement et suivre les changements nombreux et spontan\'es qui peuvent survenir au cours du développement.

Freeem~\cite{gama_greateGDD} écrit que le GDD doit non seulement décrire le corps du jeu vidéo mais son <<âme>>.
Selon lui, construire un GDD doit être un moyen de présenter le résultat attendu dans les moindres détails pour permettre aux équipes de travailler sur une idée précise.
Faire appel à des outils graphiques plutôt qu'à des textes peut être un moyen d'apporter plus de précision à une idée.
Mais le GDD doit être un moyen d'expliquer toutes les parties du jeu, le <<quoi>>, mais également le <<comment>>.

Cependant décrire l'intégralité des détails dans le GDD peut mener à des dérives décrites par Rouse~\cite{GD_theory_rouse}.
À vouloir apporter trop de détails dans un GDD, un \emph{Game designer} peut perdre énormément de temps, qui aurait pu être économisé en ayant plus de communication avec son équipe.
Trop de détails peut également apporter trop de limitations aux corps de métier plus artistiques et brider leur créativité.
Certaines parties du jeu peuvent également en obstruer d'autres : trop de détails dans la description d'une mécanique peut éclipser le fait qu'une autre mécanique n'est pas assez détaillée.
Un GDD trop complet donne également l'impression que le projet est terminé et qu'il ne nécessite plus d'ajustements, ce qui peut nuire à l'évolution du jeu et aux modifications positives qui pourraient devoir lui être apport\'ees.


\GT{Il va falloir que je relise cette derni\`ere section --- en fait,
le chapitre au complet --- car j'ai l'impression qu'il y a certaines
redondances/r\'ep\'etitions, mais aussi quelques aspects manquants
dans le fil logique.}



\section{Conclusion}
Dans ce chapitre, nous avons vu trois types de documents de \emph{Game design}~: 
\begin{itemize}
    \item Le \emph{One-Page} et le \emph{One-Sheet}, qui servent de premier aperçu du jeu.
    \item Le \emph{Five-Pager} et le \emph{Ten-Pager}, qui permettent de présenter le jeu à une équipe de développement et permettent de référencer les premières pistes de \emph{Game design}
    \item Le \emph{Game Design Document}, qui permet de regrouper tous les éléments de design pour un jeu en cours de conception ou de développement.
\end{itemize}



\gt{Les nommer ici, comme rappel/r\'esum\'e/synth\`ese}
%  
Ces documents sont les plus répandus dans le monde du développement de jeux vidéos. 

Selon le type de \emph{gameplay}, la quantité d'informations ou la durée de vie du jeu, un GDD peut avoir une taille plus ou moins importante. 
%
La structure du GDD peut différer d'un \emph{Game designer} à l'autre en fonction des habitudes de travail avec son équipe ou son expérience.
Le contenu du GDD peut aussi être sous formes diverses :  textes,  graphiques, diagrammes,  \emph{mind mapps}. 

%La structure de tels documents n'est pas fixe et leur rédaction peut \^etre effectuée en suivant des bonnes pratiques décrites par des \emph{Game designer} expérimentés et trouvent réellement leur forme par la manière dont un \emph{Game designer} souhaite partager sa vision du futur jeu.

\gt{Derni\`ere phrase \`a relire!!!}
\eh{Je l'ai modifiée et reformulée pour celle en dessous.}

C'est pour ces raisons qu'il est compliqué d'établir un gabarit précis pour la rédaction d'un GDD.
Aucun standard n'existe afin de définir précisément l'architecture des documents de \emph{game design}.
La rédaction de ces documents se fait en suivant des bonnes pratiques décrites par des \emph{game designers} expérimentés.
Chaque \emph{game designer} peut définir son propre gabarit en fonction de ses besoins et de ses habitudes.
