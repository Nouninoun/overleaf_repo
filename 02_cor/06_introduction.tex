\begin{introduction}

%ou
%quoi
%pourquoi
%comment


Le développement de jeux vidéos est un domaine en pleine expansion.
%
Les jeux deviennent plus complexes, les projets plus compliqués, les
budgets plus importants, les d\'elais de développement plus serr\'es.
%
De nombreux défis entourent donc le développement de jeux vidéos et
les outils et méthodes pour les relever sont variés~: moteurs de jeu,
environnements de développement intégrés, outils de gestion de projet,
logiciels de création 3D, aide au développement d'intelligences
artificielles, gestion des animations, etc.

Cependant, un pan complet de la création de jeux vidéos reste encore
peu formalisé et peu outillé : le~\emph{game design}.
%
Pourtant, la réflexion sur le~\emph{game design} et la documentation des concepts d'un jeu vidéo sont des étapes cruciales.
%
Sans une solide documentation et des concepts formul\'es de fa\c{c}on claire, un
projet de développement peut être rapidement voué à l'échec.
%
En outre, un \emph{game designer} se doit d'être précis et concis dans ses
descriptions et ses communications afin de transmettre un maximum
d'informations et, surtout, afin de pouvoir dédier plus de temps à l'exploration, à
la recherche artistique et au développement d'un prototype.

\gt{Ci-bas, à deux endroits: tu utilises le terme
<<pré-conception>>. Or, dans le chapitre I, ce terme n'est pas
utilisé: tu as les termes <<conception>> et <<pré-production>>.   À
corriger pour mettre le bon terme.}

%quoi
Dans ce mémoire, nous présentons \gls{gg}, un profil~UML --- \ACOMPLETER{quelques mots qui disent c'est quoi} --- qui permet d'outiller le processus de conception d'un jeu vidéo, facilitant ainsi la documentation et la rédaction des documents descriptifs du jeu vidéo en cours d'élaboration.

\begin{comment}
\gt{Ci-bas: il vaut mieux ne pas parler <<d'acc\'el\'eration>>, car tu
n'apportes aucune preuve/justification que cela permet d'aller plus
rapidement.}
\end{comment}


%
Destiné à être utilisé dans les premières phases du développement,
\emph{Breakthrough} et Conception, le profil \emph{Game Genesis}
permet de documenter, à l'aide d'une notation graphique, les
\'el\'ements de \emph{\bf mécanique} d'un jeu --- au sens du
\emph{framework} \gls{mda}~\cite{MDA_formal}, c'est-\`a-dire,
les principaux éléments du jeu et les association et interactions
entre eux.
%
En d'autres mots, l'utilisation de \emph{Game Genesis} aide \`a
documenter le \emph{mod\`ele conceptuel} d'un jeu, les principaux
concepts qui le caract\'erisent.


\begin{comment}
%cohérence
L'utilisation d'un profil UML permet d'apporter une structure précise et un cadre de travail rigoureux lors de la conception afin d'assurer la cohérence des modèles durant toute leur durée de vie.


\GT{Ci-bas: ok d'insister sur l'aspect documentation visuelle, mais
moins pertinent d'insister sur l'aspect facilit\'e de
modification. Comme tu as pu le constater en produisant le mod\`ele de
PUBG, ce n'est pas si facile de modifier un tel mod\`ele et de le
faire \'evoluer! Donc \`a reformuler!}


%maintenance & modifications
Lors de sa conception, un jeu vidéo est destiné à évoluer rapidement et sa documentation se verra modifiée de nombreuses fois afin de toujours être la plus pertinente possible.
L'utilisation d'UML permet d'apporter un support visuel à cette documentation et permet également de rendre les modifications plus faciles en accédant aux objets concernés rapidement et efficacement. 


\GT{Ci-bas: je crois que cette partie sur le mind mapping n'est pas
utile/n\'ecessaire, car elle arrive de fa\c{c}on un peu abrupte,
inattendue, comme un <<cheveu sur la soupe>> --- pourquoi
sp\'ecifiquement le mind mapping, et pas d'autres approches? Je
sugg\`ere donc de la supprimer.}

%pérennité
Effectuer des \emph{mind mapping} est un moyen simple et efficace permettant d'illustrer des idées de manière visuelle.
Cependant, le manque de rigueur (modifications erratiques, contenu incohérents, manque de structure) et les supports utilisés (feuille de papier ou tableau) lors d'un travail de \emph{mind mapping} n'assurent pas du tout la pérennité des informations stockées.
Bien qu'il existe des logiciels de \emph{mind-mapping} qui conservent les informations sous la forme d'un fichier XML, il n'existe pas réellement de <<~standard~>> définissant de manière formelle un langage de \emph{mind-mapping}.
De plus, chaque outil de \emph{mind-mapping} possède ses propres notations, méthodes et représentations.

De par sa structure standardis\'ee et précise, la notation UML permet de structurer les concepts, de les stocker sous un format précis et g\'en\'eralement portable.
Ceci permet un stockage beaucoup plus pérenne des informations contenues dans un modèle.
%
En outre,
UML étant un langage de modélisation largement utilisé et approuvé dans le domaine informatique, il est possible d'utiliser de nombreux outils existants, d'assurer son intégration et sa réutilisation dans différents cadres et sous différentes formes.



\end{comment}


\gt{Ci-bas: \`a compl\'eter en d\'ecrivant en une courte phrase ce que
pr\'esente le chapitre en question.}

Dans ses grandes lignes, ce m\'emoire est organis\'e comme suit.
Le chapitre~\ref{chap.dev_jv} pr\'esente les principaux concepts
li\'es au d\'eveloppement de jeux vid\'eos, notamment la notion de mécanique de jeu.
Nous y traitons également des étapes de création d'un jeu vidéo, des notions d'exploration et d'exploitation, des moteurs de jeux et des genres vidéo-ludiques. 

\gt{Ci-bas: le terme <<Executive Summary>> n'est pas du tout utilis\'e
dans le chapitre sur le GDD. Donc, soit tu l'utilises dans ce
chapitre, soit tu ne l'utilises pas ci-bas.}

Le chapitre~\ref{chap.design_doc} pr\'esente diff\'erentes approches
propos\'ees dans la litt\'erature pour documenter un \emph{game design}.
Nous y présentons le contenu d'un \emph{Concept Document}, d'un \emph{five (ou ten) pager} et d'un \emph{Game Design Document}.

Le chapitre~\ref{chap.MDA} pr\'esente le \emph{framework} MDA de Hunicke, Leblanc et Zubek~\cite{MDA_formal}, qui découpe un jeu en trois aspects distincts : les éléments de \emph{Mechanics} (données et algorithmes), les éléments de \emph{Dynamics} (comportement à l'exécution des éléments de \emph{Mechanics}) et les éléments d'\emph{Aesthetics} (émotions générées par l'expérience de jeu). Nous y discutons également de certaines limites du \emph{framework MDA} et de quelques propositions alternatives.

Le chapitre~\ref{chap.profils-UML} pr\'esente la notion de <<profil>>
en UML, qui permet d'étendre les mécanismes d'UML afin de l'adapter à un domaine d'activité ou à une plateforme de développement particulière.

Le chapitre~\ref{chap.game-genesis} pr\'esente \emph{Game Genesis}, le
profil UML que nous avons d\'evelopp\'e, pour aider \`a la
mod\'elisation des \'el\'ements de m\'ecanique d'un jeu vid\'eo.

Finalement, le chapitre~\ref{chap.pubg} pr\'esente un exemple d'application de \emph{Game Genesis} en décrivant les éléments de~\emph{Mechanics} du jeu \emph{PlayerUnknown's BattleGrounds}.




\end{introduction}

