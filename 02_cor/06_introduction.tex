\begin{introduction}


De nombreux défis entourent le développement de jeux vidéos et les outils et méthodes pour y parvenir sont variées.
Cependant, il devient de plus en plus important de chercher des méthodes afin d'accélérer le développement, sans pour autant impacter la qualité du contenu final.
Un \emph{Game designer} se doit d'être précis et concis dans ses descriptions et ses communications afin de transmettre un maximum d'informations et de pouvoir dédier plus de temps à l'exploration, à la recherche artistique et au développement d'un prototype. 

%quoi
Dans ce mémoire, nous allons présenter \gls{gg} (cf.~Section~\ref{game-genesis.sect}), un profil UML (cf.~Section~\ref{profils-UML.sect}) qui permet d'outiller le processus de pré-conception d'un jeu vidéo, facilitant ainsi la documentation et la rédaction des documents descriptifs du jeu vidéo en cours d'élaboration.

%rapidité
Destiné à être utilisé dans les premières phases du développement, \emph{Breaktrough} et Pré-conception~\ref{prob.etapes}, ce profil UML permet l'accélération de la documentation à travers un outil visuel représentant les mécaniques et les objets du jeu.
Ceci permet de produire un prototype jouable plus rapidement.

%cohérence
L'utilisation d'un profil UML permet d'apporter une structure précise et un cadre de travail rigoureux lors de la conception afin d'assurer la cohérence des modèles durant toute leur durée de vie.

%maintenance & modifications
Lors de sa conception, un jeu vidéo est destiné à évoluer rapidement et sa documentation se verra modifiée de nombreuses fois afin de toujours être la plus pertinente possible.
Le standard UML --- \gls{uml} --- permet d'apporter un support visuel à cette documentation et permet également de rendre les modifications plus faciles en accédant aux objets concernés rapidement et efficacement. 

%pérennité
Effectuer des \emph{mind mapping} est un moyen simple et efficace permettant d'illustrer des idées de manière visuelle.
Cependant, le manque de rigueur (modifications erratiques, contenu incohérents, manque de structure) et les supports utilisés (feuille de papier ou tableau) lors d'un travail de \emph{mind mapping} n'assurent pas du tout la pérennité des informations stockées.
Bien qu'il existe des logiciels de \emph{mind-mapping} qui conservent les informations sous la forme d'un fichier XML, il n'existe pas réellement de <<~standard~>> définissant de manière formelle un langage de \emph{mind-mapping}.
De plus, chaque outil de \emph{mind-mapping} possède ses propres notations, méthodes et représentations.

De par sa structure imposée et précise, le standard UML permet de structurer les données, de les stocker sous un format précis et de les maintenir facilement en communiquant sur les modifications effectuées.
Ceci permet un stockage beaucoup plus pérenne des informations contenues dans le modèle.

\gt{Il faudra r\'eviser ce qui pr\'ec\`ede: on peut aussi utiliser un
outil logiciel pour cr\'eation de Mind Maps, par ex., FreeMind, qui
conserve le MindMap sous forme XML.  N'en reste pas moins un
d\'esavantage: ce n'est pas un standard, i.e., chaque outil de
MindMapping a son propre format, ses propres variantes et
repr\'esentations, etc. J'ai l'impression que c'est plus de ce cot\'e
qu'il faut argumenter, que sur l'aspect papier/photos des mind maps.}


%réutilisation
UML étant un langage de modélisation largement utilisé et approuvé dans le domaine informatique, il est possible d'utiliser de nombreux outils existants, d'assurer son intégration et sa réutilisation dans différents cadres et sous différentes formes.

%gdd
Un modèle UML pourra également servir de support à la rédaction d'un \gls{gdd}~\ref{sect.GDD}, en organisant les mécaniques de jeu sous forme de catégories réutilisables.
Le GDD est la <<~bible du design~>>~\cite{GD_foundations_pedersen} du jeu.
Il permet de définir tout le contenu du jeu et toutes les informations nécessaires à son développement, et ce sur toute sa durée de vie. 

%generation de code
UML étant reconnu comme langage de modélisation de logiciels, il peut également être possible d'utiliser les modèles créés en pré-conception afin de faciliter le développement informatique du jeu.
Le modèle peut permettre de créer une structure de projet ou du pseudo code pouvant servir de squelette en début de développement.
\end{introduction}

