%\documentclass[12pt,twoside]{memoireuqam1.3}
\documentclass[12pt]{memoireuqam1.3}
% Si vous souhaitez en recto-verso
\usepackage{graphicx}% Pour les figures
%\usepackage[square,numbers]{natbib}
\usepackage[numbers]{natbib}
\usepackage[french]{babel}
\usepackage[utf8]{inputenc} % Pour utiliser les caractères accentués
\usepackage[T1]{fontenc}
\usepackage{url}
\usepackage{framed}



% OOOOOOOOOOOOOOOOOOOOOOOOOOOOOOOOOOOOOOOOOOOOOOOO
%
% Pour accelerer le traitement du fichier pendant que je lis/relis/revise. (GT)

%\includeonly{02_cor/10_UMLprofile}
%\includeonly{02_cor/11_GG}
%\includeonly{02_cor/12_application}

% OOOOOOOOOOOOOOOOOOOOOOOOOOOOOOOOOOOOOOOOOOOOOOOO


%----------------------------------------------------------------

%Autres packages et commandes utiles
%----------------------------------------------------------------
\usepackage{amsmath,amsthm,amssymb,amsfonts}	% Pour pouvoir inclure certains symboles et environnements mathématiques
\usepackage{enumerate,enumitem} % Pour mieux gérer la commande enumerate dans les sections
\usepackage{rotating,graphicx,multicol}	% Pour inclure des images et les colonnes
\usepackage{array}
\usepackage{color}	% Pour inclure du texte en couleur
\usepackage{units}	% Pour pouvoir tapper les unités correctement
\usepackage{pgf,tikz}	% Utilisation du module tikz, qui permet de tracer des belles images
\usetikzlibrary{arrows} % Quand on exporte une image GeoGebra, on a besoin de préciser cela
\usepackage{hyperref}	% Pour include des liens dans le document
\newcommand{\N}{\mathbb{N}}	% Commande personnelle, plus rapide pour taper les ensembles
\newcommand{\Z}{\mathbb{Z}}	% Commande personnelle, plus rapide pour taper les ensembles
\newcommand{\R}{\mathbb{R}}	% Commande personnelle, plus rapide pour taper les ensembles
\usepackage{cprotect}	% Pour pouvoir personaliser la légende des figures
\usepackage{float} % Pour pouvoir utiliser d'autres placements de figures tels que H (HERE forcé h!)
\usepackage{subcaption}
\usepackage{comment}
\usepackage{qtree}
\usepackage{pdflscape}
\usepackage{adjustbox}
\usepackage[edges]{forest}
\setcounter{tocdepth}{3}
\setcounter{secnumdepth}{3}

%%%%%%%%%%%%%%%%%%%%%%%%%%%%%%%%%%%
% Items, and Enum

\newcommand{\GT}[1]{{{\textcolor{red}{\footnotesize [[(Guy T.) #1]]}}}}
\newcommand{\gt}[1]{}

%%%%%%%%%%%%%%%%%%%%%%%%%%%%%%%%%%%

%Création du glossary (liste des acronymes)
\usepackage[toc,section=chapter,acronym,nomain]{glossaries}
\makeglossaries

% Marquage des chapitres dans la liste des figures
% \usepackage{etoolbox}
% \preto\figure{%
%   \ifnum\value{figure}=0\addtocontents{lof}{\par\bfseries Chapitre~\thechapter\vskip10pt}\fi
% }
%
% GT: Pas necessaire, d'autant plus que le numeros de chapitre sont
% indiques en chiffres arabes, alors qu'ils devraient l'etre en
% chiffres romains majuscules. J'ai essaye de les mettre en romains,
% mais sans succces :( 


\begin{document}
%%%%%%%%%%%%%%%%%%%%
% Pour la page titre
%%%%%%%%%%%%%%%%%%%%
\title{\emph{Game Genesis} \hspace{20cm} Un profil UML pour la rédaction d'un \emph{Game Design Document}}
% Votre nom complet tel qu'il apparaît à votre dossier du registrariat de l'UQAM
\author{Ellen Haas}
% Année et mois courant sauf si spécifié autrement pas \degreemonth et \degreeyear
%\degreemonth{mois du dépôt}
%\degreeyear{année du dépôt}
\uqammemoire %% ou \uqamthese ou \uqamrapport
\matiere{informatique}
\thispagestyle{empty}        % La page titre n'est pas numérotée
\maketitle



%%%%%%%%%%%%%%%%%%%%
% Page préliminaires
%%%%%%%%%%%%%%%%%%%%
%si vous voulez qu'apparaisse le titre RÉFÉRENCES plutôt que BIBLIOGRAPHIE
%\renewcommand \bibname{R\'EF\'ERENCES}% FACULTATIF
% (GT) D'apres le guide presentation des memoires, c'est BIBLIOGRAPHIE qui doit etre utilise.
\renewcommand \bibname{BIBLIOGRAPHIE}
\renewcommand \listfigurename{LISTE DES FIGURES}
\renewcommand \appendixname{ANNEXE} % En francais, pour un memoire, c'est plutot annexe!
\renewcommand \figurename{Figure}
\renewcommand \tablename{Tableau}

\pagenumbering{roman}  %% numérotation des pages liminaires en chiffres romains
\addtocounter{page}{1} %% les remerciements commencent à la page ii
\chapter*{Remerciements}


Tout d'abord je souhaite remercier mon directeur de recherche, Guy Tremblay.
Pour sa présence, son temps ainsi que sa patience tout au long de ma maîtrise.
Je souhaite aussi le remercier pour la confiance qu'il m'a accordée lorsque je lui ai proposé un sujet de recherche de ma propre initiative.
Ses conseils et son expérience m'ont accompagnée tout au long de ma recherche et de ma rédaction.


Je souhaite également remercier mon conjoint, Jehan.
Il m'a encouragée dans les moments de travail intense, épaulée durant les moments compliqués et m'a secouée dans les moments où je ne pensais plus y arriver. 
Ses paroles rassurantes, ses encouragements bruyants et la confiance inébranlable qu'il a exprimée tout au long de cette maîtrise ont été pour moi le socle solide qui m'a permis de continuer.


Je souhaite également remercier des professeurs qui ont joué un rôle important dans ma vie.
M.~Flieg qui a su rallumer en moi l'envie d'étudier et de découvrir de nouvelles choses en appréhendant le monde sous un angle différent.
M.~Froeliger sans qui j'aurais abandonné l'idée de faire mon échange à l'UQAM avant même d'être venue.


Je remercie également mes amis.
Qu'ils soient tout près ou très loin.
Que je les côtoie toutes les semaines ou que je les voie rarement.
Ils sont nombreux, je ne les citerai donc pas, mais ils se reconnaîtront.


Enfin je tiens à remercier ma famille.
Mon père tout d'abord, sans qui rien n'aurait été possible.
Pour la confiance qu'il m'accorde, pour la patience dont il a toujours fait preuve et pour son soutien au quotidien depuis de nombreuses années.
Je remercie également ma grande soeur qui, même si elle est loin, a toujours un oeil sur ce que je fais.

%\chapter*{Dédicace}
\begin{flushright}
\begin{small}

Je dédie ce mémoire à mon fils, Nahel. 
\\Pour que tu puisses avoir la même chance que moi
\\de faire tes propres choix dans la vie.
\\Pour que je puisse te soutenir et t'épauler 
\\dans le chemin que tu suivras 
\\comme ton grand père l'a fait pour moi.

\end{small}
\end{flushright}

%\include{01_lim/03_avant-propos}

\tableofcontents % Pour générer la table des matières
\listoffigures % Pour générer la liste des figures
\listoftables % Pour générer la liste des tableaux

\newacronym{dlc}{DLC}{Contenu téléchargeable (Downloadable Content)}
\printglossary[title=LISTE DES ABRÉVIATIONS\, DES SIGLES ET DES ACRONYMES]
\begin{abstract}
Le développement de jeux vidéos est un domaine en pleine expansion depuis plusieurs années.
Les jeux deviennent de plus en plus complexes et de plus en plus importants en taille.
Les projets deviennent beaucoup plus compliqués, les budgets plus importants et le temps de développement de plus en plus réduit.
Afin d'accélérer toutes les étapes du développement d'un jeu vidéo de nombreux outils émergent dans ce domaine.
Les moteurs de jeu mettent à disposition des environnement clé en main afin d'aider les équipes de développement ils incluent : 
les environnements de développement intégrés, les outils de gestion de projet, les logiciels de création 3D, les environnements de développement d'intelligences artificielles, la gestion des animations, etc.
Cependant un pan complet de la création de jeux vidéos reste encore peu formalisé et peu outillé : le~\emph{game design}.
Pourtant la réflexion sur le~\emph{game design} et la documentation des concepts d'un jeu vidéo sont des étapes cruciales du développement de jeu vidéo.
Sans une solide documentation et des concepts clairement énoncés un projet de développement peut être rapidement voué à l'échec.
Nous avons recherché les bonnes pratiques de rédaction d'un~\emph{Game Design Document} ainsi que les outils actuellement utilisés pour les étapes de~\emph{game design} et de pré-conception.
Dans ce mémoire nous allons proposer~\emph{Game Genesis}, un profil UML pour faciliter la rédaction d'un~\emph{Game Design Document}.
UML introduit : un langage formel, une structure précise, des outils performants et un support de communication efficace.
Le domaine du développement de jeux vidéos est cependant très vaste. 
La taille des équipes de développement, les genres de jeux vidéos, les types de~\emph{gameplay}, les spécificités de~\emph{game design} entraînent une difficulté à anticiper tous les éléments nécessaires à la rédaction de~\emph{Game Design Document}.
La liste des stéréotypes définis dans~\emph{Game Genesis} est donc non exhaustive et les éléments répertoriés sont assez généraux afin de ne pas représenter d'obstacles à la représentation de certains types de jeux vidéos.
Afin de structurer ce profil nous allons faire usage du~\emph{Framework MDA} qui sépare le~\emph{game design} en trois catégories d'éléments :~\emph{Mechanics},~\emph{Dynamics},~\emph{Aesthetics}.
Avec~\emph{Game Genesis} nous proposons une structure pré-existante afin de représenter les~\emph{Mechanics} d'un jeu vidéo.
Nous avons pu établir une liste d'éléments de~\emph{Mechanics} permettant de représenter les jeux vidéos.
L'utilisation de~\emph{Game Genesis} 


MOTS CLÉS :~\emph{Game Design}, jeux vidéos,~\emph{Game Design Document},~\emph{Framework MDA}, profil UML.

\end{abstract}







% Utilisez l'environnement  abstract pour rédiger votre résumé


%%%%%%%%%%%%%%%%%%%%
% Document principal
%%%%%%%%%%%%%%%%%%%%

\begin{introduction}

Le développement de jeux vidéos \\
Les métiers impactés \\
Les outils de travail \\
Etapes de développement d'un jeu vidéo \\
Focus sur Breakthrough et Conception \\
Les livrables du développement \\
\end{introduction}


% Utilisez l'environnement  introduction pour rédiger votre introduction
\chapter{Le développement de jeux vidéos}
 \textcolor{red}{Trouver une manière d'introduire le chapitre}

\section{Les étapes de création d'un jeu vid\'eo}
%Conférence Ubisoft a l'UQAM
\begin{figure}[H]
    \centering
    \includegraphics[width=14cm]{10_img/production_stages.png} 
    \caption{Étapes de création d'un jeu vidéo.}
    \label{fig.etapes}
\end{figure}

Les étapes de développement d'un jeu vidéo ne sont pas immuables et dépendent de l'entreprise, du domaine d'activité ou des collaborateurs impliqués dans le projet. La figure \ref{fig.etapes} présente une liste non exhaustive de ces étapes, telle que présentée par Mathieu Nayrolles lors d’un séminaire au LATECE (Laboratoire de recherche sur les Technologies du Commerce Électronique) de l’UQAM, le 10 avril 2019.



\subsection{\emph{Breakthrough}}

Une étape de \emph{Breakthrough} permet de réunir une équipe afin d'effectuer des recherches et des explorations sur des nouvelles mécaniques ou des nouvelles technologies afin de créer du nouveau contenu. Cette étape est optionnelle dans un projet.
Deux exemples :
\begin{itemize}
    \item Une percée technologique, par ex., Google lance Stavia, une plateforme en ligne de jeux vidéos sous forme de catalogue et jouable à 100\% en ligne sans installation en local;
    \item Une percée de Gameplay, par ex., naissance du mode Battle Royale.
\end{itemize}

\subsection{Conception}
%Conception et non concept car c'est là que les documents sont rédigé, on parle non seulement du QUOI mais aussi du COMMENT les documents sont déjà explicites sur le COMMENT le jeu fonctionnera et sera développé

Un document de conception est défini afin d'identifier l'environnement, la faisabilité et l'intérêt commercial du jeu. C'est durant cette étape que les \emph{Game designers} définissent et précisent l'univers, les mécaniques et le déroulement du jeu vidéo en question. Cette étape est majoritairement gérée par les \emph{Game designers} appuyés par les équipes des autres corps de métier. Dans des projets à financements externes, cette étape est cruciale: elle permettra de présenter le projet à un studio avec une première maquette présentant les fondements du jeu.

\subsection{Pré-production}
Durant cette étape, des prototypes sont développés afin de créer une version minimale du jeu. Ces prototypes permettent d'avoir un aperçu jouable des concepts définis durant la phase précédente. 

Un prototype est une coupe verticale de tout le système qui permet de valider ou redéfinir les concepts. Une fois le design bien défini, le prototype plus proche de ce que donnerait le jeu final et la faisabilité du projet confirmée, il est alors possible de rechercher les financements et les ressources nécessaires à la production. Cette étape est gérée par tous les corps de métier dans un studio de développement, tous les aspects du jeu devant être représentés afin de montrer tout le potentiel du prototype.

\subsection{Production}
Une fois les fonds levés et les ressources humaines attitrées au projet, il est possible de procéder à la production du jeu vidéo. Tous les corps de métier sont représentés et le jeu est développé sous tous ses aspects et dans son intégralité. La plupart du temps le développement est découpé en plusieurs itérations. Chacune d'entre elles permet de vérifier que le jeu respecte bien les concepts définis plus tôt. C'est également à ce moment là que les dernières modifications sont apportées aux concepts afin qu'ils respectent la vision du \emph{Game designer} et génèrent la bonne expérience. Dans le cas o\`u le projet rencontre des contraintes supplémentaires (temps, argent, plateforme, etc.) les concepts peuvent également être revus. Généralement, durant cette étape, le marketing et la publicité autour du jeu commencent à prendre place afin d'informer le public et d'estimer l'impact que peut avoir le jeu.


\subsection{\emph{World Simulation}}
Lors de la \emph{World simulation}, tous les éléments du jeu sont testés et passés au crible. On vérifie que les éléments de jeu sont correctement modélisés, que les sons correspondent bien aux éléments, que les personnages correspondent à ceux décrits dans les documents de conception, etc. Plusieurs questions se posent à ce moment là. 
\begin{itemize}
    \item Est-ce que les éléments interagissent bien entre eux ?
    \item Est-ce que l'univers de jeu est cohérent de bout en bout ?
    \item Est-ce que le \emph{gameplay} est fluide et intuitif ?
    \item Est-ce que l'environnement de jeu est réellement tel que décrit par le \emph{Game designer}?
    \item Est-ce que la bande sonore ou la modélisation graphique génèrent bien les émotions attendues chez le joueur ?
\end{itemize}


\subsection{\emph{Operate}}
Le jeu est maintenant produit et commercialisé. Une quantité importante de données est ainsi générée. Des bogues peuvent être remontés et corrigés dans des cas de figure particuliers ou inédits non couverts par l'étape de \emph{World simulation}. Des ajustements mineurs peuvent être faits en fonction des besoins ou des demandes des joueurs. Le jeu prend alors toute sa dimension et toute sa vie à travers les joueurs.

Une fois le jeu bien en place et les étapes de corrections et ajustements passées, il est possible d'intégrer du nouveau contenu au jeu en repassant par les étapes précédentes. Ce nouveau contenu est généralement intégré au jeu sous forme de mises à jours (gratuites) ou de \gls{dlc} (payantes).



\section{Exploitation et exploration}
\subsection{Exploitation}
Le travail d'exploitation consiste à produire une suite ou un nouveau jeu en utilisant des technologies (moteur, plateformes, etc) ou un \emph{gameplay} déjà existant afin de recréer un jeu ou produire du contenu additionnel. Ceci peut être fait dans un but de fidéliser une clientèle déjà existante en ajoutant du contenu additionnel à un jeu, à offrir une expérience similaire avec des technologie plus récentes (ex : FIFA) ou à offrir une suite à un jeu ayant déjà connu du succès (ex : Dark Souls).

L'exploitation est une part importante du travail d'un studio de développement. De nombreux jeux vidéos récents sont fondés sur de l'exploitation de jeux précédents, autant au niveau du \emph{gameplay} qu'au niveau des concepts fondamentaux de jeux précédents. C'est le cas de grosses productions de franchises comme les jeux de EA sports (FIFA, NHL, NBA Live, Madden), les jeux d'action \emph{role-play} de FromSoftware (série des Dark Souls), les jeux d'action aventure de Rockstar (série des GTA) ou les jeux de simulation de Maxi/EA Games (série les Sims). 

\subsection{Exploration}
L'exploration, ou l'innovation, dans le monde du jeu vidéo est essentielle au développement de nouveaux concepts de \emph{gameplay} mais également au développement de nouvelles technologies. La nouveauté est un enjeu essentiel afin d'attirer toujours plus de joueurs. L'investissement dans l'exploration est donc très important pour un studio de développement. De la recherche de nouveaux concepts de jeux, de nouveaux types de \emph{gameplay}, de nouvelles technologies à intégrer ou de la création de nouveaux moteurs de jeux, l'exploration est devenue un facteur essentiel au domaine du jeu vidéo et à son expansion.


\subsection{L'équilibre entre exploitation et exploration}
Dans leur article, Parmentier \emph{et al.} \cite{ParmentierGuy2009Iecd} explorent la capacité des studios à concilier ces deux activités. Ils présentent les enjeux de chacune d'entre elles et leur importance dans le domaine.

Il est nécessaire pour les studios de développement de jeux vidéos de trouver le bon équilibre entre exploitation et exploration. L'exploitation est le développement d'un jeu sur des mécanismes déjà existants où les règles sont déjà établies par un autre jeu ou un opus précédent. L'exploration est la découverte de nouvelles mécaniques de \emph{gameplay} ou la création de nouveaux outils de développement de jeux (comme un moteur de jeu). Le but est d'offrir aux clients des articles de qualité et attractifs. Cet équilibre est précaire et il est difficile pour un studio de développement d'investir sur les deux domaines à la fois. 



\section{Les moteurs de jeux}
Un moteur de jeu (\emph{Game engine}) est, typiquement, une suite logicielle contenant un framework de mécaniques de jeu permettant d'accélérer le développement d'un jeu vidéo. Il peut inclure une ou plusieurs facettes du développement du jeu allant de la physique, aux graphismes, aux sons, aux calculs, à la gestion des périphériques d'entrée/sortie jusqu'à la gestion automatique de l'intelligence artificielle. Voici une liste de certains des moteurs de jeu les plus connus accompagnés des jeux qui en font usage : 
\begin{itemize}
    \item \emph{Unreal Engine} \cite{UnrealEngine} développé par Epic Games : Fortnite, Outlast 2, Dragon Ball Fighter Z, Days Gone.
    \item \emph{Unity Engine} \cite{UnityEngine} développé par Unity Technologies : 7 Days to Die, Cuphead, Ori and the Blind Forest, Pokemon Go.
    \item \emph{CryEngine} \cite{CryEngine} développé par Crytek : Far Cry, Crysis 3, Deceit, Mavericks.
    \item \emph{Frostbite} \cite{FrostbiteEngine} développé par Dice (EA) : Battlefield V, Anthem, FIFA, Need for Speed.
\end{itemize}

Chaque moteur de jeu présente des avantages et des inconvénients en fonction du type de jeu que l'on souhaite développer. Certains moteurs sont axés sur un type de jeu ou une plateforme en particulier afin d'être plus efficaces. 
L'innovation dans les moteurs de jeu est aussi essentielle au développement de nouveaux jeux vidéos. 
Par exemple, un moteur de jeu plus récent pourra intégrer des éléments plus récents comme des graphismes plus réalistes ou détaillés, ou des intelligences artificielles plus évoluées.

\section{Les types de \emph{Gameplay}}
L'exploration peut également consister en la création d'un nouveau mécanisme de \emph{gameplay}. Ce genre d'innovation est plus facilement repérable par les joueurs et plus marquante en ce qui a trait \`a l'expérience de jeu. 

Voici une liste non exhaustive des principaux types de \emph{gameplay} présents dans les jeux vidéos :
\begin{itemize}
    \item MMORPG (\emph{Massive Multiplayer Online Role-Playing Games}) : jeu massivement multijoueur en ligne mettant en scène un jeu de rôle avec différents objectifs à remplir : \emph{leveling}, histoire principale vs.\ secondaire, développement social pour atteindre ces objectifs sous forme de guilde, etc. (ex. : World of WarCraft, Black Desert Online).
    \item \emph{Survival} : le joueur doit survivre aux événements présents dans le jeu. Il peut devoir subvenir à des besoins vitaux, construire de nouveau objets, ou survivre aux autres joueurs présents (ex. : Rust, Ark).
    \item Plateformes : un joueur contrôle un personnage qui se déplace dans un environnement de plateformes et doit avancer tout au long du niveau pour le compl\'eter (ex. : Mario, Donkey Kong).
    \item Simulation de vie : le joueur simule un environnement de vie plus ou moins réaliste en fonction des objectifs du jeu. La simulation peut s'appliquer à un personnage ou à une ville entière. (ex. : Les Sims, SimCity).
    \item FPS (\emph{First Person Shooter}) : le joueur, seul ou en équipe, doit battre des ennemis (IA ou autres joueurs) à l'aide d'armes et d'équipements de combat (ex. : Call of Duty, Halo).
    \item \emph{Beat-em up} : le joueur fait face à des vagues d'ennemis toujours plus fortes (ex. : Bayonnetta, God of War).
    \item RTS (\emph{Real Time Strategy}) : des joueurs se font face dans un jeu o\`u la gestion d'économie, de troupes et de population est omniprésente afin de battre les autres (Age of Empire, Starcraft).
    \item 4X (\emph{eXplore, eXpand, eXploit, and eXterminate}) : proche du RTS, ce type de \emph{gameplay} est fondé sur une gestion pointue de ressources et de populations afin de pouvoir battre les autres joueurs sous différents aspects et avec différents objectifs de victoire (population maximale, évolution de la société, critères financiers, etc.) (ex. : Civilization, Stellaris).
    \item MOBA (\emph{Multiplayer Online Battle Arena}) : c'est un type de \emph{gameplay} o\`u le jeu d'action rencontre le RTS. Plusieurs équipes de joueurs sont téléportées sur un territoire, chaque joueur contrôle un personnage et les joueurs doivent détruire la base de l'équipe adverse (ex. : League of Legends, DOTA).
    \item \emph{Battle Royal} : plusieurs dizaines de joueurs sont parachutés sur un territoire o\`u ils trouvent des armes et doivent s'entretuer~; le dernier survivant est déclaré vainqueur (ex. : Fortnite, PUBG).
\end{itemize}

Les types de \emph{gameplay} évoluent beaucoup. De nouveaux \emph{gameplay}s apparaissent grâce aux recherches exploratoires. Certains modes de jeux à succès deviennent des catégories à part entière. Il est possible de combiner plusieurs modes de jeu afin de créer une nouvelle expérience. Ces divers types de \emph{gameplay} sont classifiés en fonction du type de monde, des objectifs de jeu, des actions nécessaires, etc. Haitz et Law \cite{HeintzStephanie2015TGGM} ont mis en place une cartographie des genres afin de classifier les différents types de jeux en se référant à des caractéristiques précises des jeux et de leur \emph{gameplay}. Cependant, il est difficile d'arriver à classifier tous les jeux tellement les genres sont nombreux et entrecoupés. C'est pour cela que la plupart des jeux sont classifiés dans des catégories larges et sont ensuite différenciés par leurs diverses caractéristiques.



\section{Objectifs de notre travail de recherche}

\GT{On reverra cette section ultérieurement. Une partie de cela
devrait \^etre dans l'introduction.}

De nombreux défis entourent le développement de jeux vidéos et les outils et méthodes pour y parvenir sont déjà nombreux. Cependant, il devient de plus en plus important de chercher des méthodes afin d'accélérer le développement, sans pour autant impacter la qualité du contenu final. Un \emph{Game designer} se doit d'être précis et concis dans ses descriptions et ses communications afin de communiquer un maximum d'informations et de pouvoir dédier plus de temps à l'exploration, à la recherche artistique et au développement d'un prototype. 



%quoi
Dans ce mémoire, nous allons présenter \gls{gg}, un profil UML \cite{ReferenceBibliographique} qui permet d'outiller le processus de pré-conception d'un jeu vidéo, facilitant ainsi la documentation et la rédaction des documents descriptifs du jeu vidéo en cours d'élaboration.

%rapidité
Destiné à être utilisé dans les premières phases du développement, \emph{Breaktrough} et Pré-conception, ce profil UML permet l'accélération de la documentation à travers un outil visuel représentant les mécaniques et les objets du jeu. Ceci permet de produire un prototype jouable plus rapidement.

%cohérence
L'utilisation d'un profil UML permet d'apporter une structure précise et un cadre de travail rigoureux lors de la conception afin d'assurer la cohérence des modèles durant toute leur durée de vie.

%maintenance & modifications
Lors de sa conception, un jeu vidéo est destiné à évoluer rapidement et sa documentation se verra modifiée de nombreuses fois afin de toujours être la plus pertinente possible. UML permet d'apporter un support visuel à cette documentation et permet également de rendre les modifications plus faciles en accédant aux objets concernés rapidement et efficacement. 

%pérennité
Effectuer des \emph{mind mapping} est un moyen simple et efficace permettant d'illustrer des idées de manière visuelle. Cependant, le manque de rigueur (modifications erratiques, contenu incohérents, manque de structure) et les supports utilisés (feuille de papier ou tableau) lors d'un travail de \emph{mind mapping} n'assurent pas du tout la pérennité des informations stockées. Il est en effet compliqué de modifier une photographie d'un \emph{mind map} lors d'une réunion s'étant tenue plusieurs semaines auparavant. De par sa structure imposée et précise le standard UML permet de structurer les données, de les stocker sous un format précis et de les maintenir facilement en communiquant sur les modifications effectuées. Ceci permet un stockage beaucoup plus pérenne des informations contenues dans le modèle.

%réutilisation
UML étant un langage de modélisation largement utilisé et approuvé dans le domaine informatique, il est possible d'utiliser de nombreux outils d\'ej\`a existants, d'assurer son intégration et sa réutilisation dans différents cadres et sous différentes formes.

%gdd
Un modèle UML pourra également servir de support à la rédaction d'un \gls{gdd} en organisant les mécaniques de jeu sous forme de catégories réutilisables. Le GDD est la \guillemotleft bible du design \guillemotright \cite{GD_foundations_pedersen} du jeu. Il permet de définir tout le contenu du jeu et toutes les informations nécessaires à son développement sur toute sa durée de vie. 

%generation de code
UML étant reconnu comme un langage de modélisation de logiciels, il peut également être possible d'utiliser les modèles créés en pré-conception afin de faciliter le développement informatique du jeu vidéo. Le modèle peut permettre de créer une structure de projet ou du pseudo code pouvant servir de squelette en début de développement.



\chapter{Les documents de \emph{Game design}}

\gt{Limiter forme <<passive>>.}

Dans ce chapitre, nous allons pr\'esenter plusieurs documents de \emph{Game design} qui sont présents dans un processus de développement de jeu vidéo.
Ces documents peuvent être rédigés l'un à la suite de l'autre, ou indépendamment l'un de l'autre, et ils ne sont pas obligatoires dans chaque projet de jeu vidéo.


\gt{Je ne comprends pas <<tr\`es discut\'es>>!?}

Les documents de \emph{Game design} sont nombreux et leurs formes ainsi que leurs contenus sont souvent remis en question dans la littérature.
Selon le niveau d'avancement du développement, certains sont plus utiles que d'autres mais la plupart des acteurs du domaine sont d'accord pour avancer qu'il n'existe pas de gabarit fixe pour un document de \emph{Game design}~\cite{GD_theory_rouse}.
En fait, un gabarit trop sp\'ecifique et précis serait une erreur à ne pas commettre, car il pourrait entraîner des difficultés de rédaction et d'adaptation face aux divers types de jeux que l'on peut rencontrer.
Cependant, certains auteurs présentent des lignes directrices, un peu comme une recette de cuisine~\cite{LevelUpRogers2014}, qui peuvent être suivies afin que le GDD soit complet et respecte une certaines forme. 

Dans sa présentation, Librande~\cite{onepage_librande} montre des exemples de livrables de game design de différentes sortes.
Il souligne que les documents de design lourds sont des sources d'informations importantes, réunissant tout le design dans un seul document et la création du document aide grandement à designer le jeu par la suite.
Cependant ces documents sont compliqués à maintenir, mettre à jours et à parcourir pour trouver une information précise.


\gt{Qui \c{c}a <<Il>>? Librande?}
\gt{Je ne comprends pas <<la contribution d'\'equipe>>?!}

Il présente également les <<\emph{design wikis}>> qui apportent beaucoup de points positifs au design.
Il est possible d'y avoir accès à n'importe quel endroit tant qu'une connexion est disponible.
Les mises à jours sont simplifiées car la recherche l'est également : il devient alors possible de modifier des éléments directement au cours d'une réunion.
Un \emph{wiki} permet aussi la contribution de tous les membres de l'équipe de développement aux différents éléments car la modification est possible par de nombreux utilisateurs.
Il est possible d'éditer les éléments par article et donc de ne pas avoir à prendre connaissance d'éléments non reliés au travail actuellement effectué.
Il est facile de maintenir un historique des versions et des modifications pour justifier les modifications et les garder en mémoire.
Cependant, un \emph{wiki} requiert une maintenance constante et donc, souvent, une personne dédiée à sa maintenance.
Les relations entre les éléments ne sont pas mises en avant et il est compliqué de mettre ensemble différentes sortes de représentation sous la forme de textes/images.
Autres d\'esavantages~: les images doivent être traitées à l'extérieur du \emph{wiki} avant d'y être réintégrées, et l'impression d'un article \emph{wiki} n'est pas toujours adaptée à une lecture rapide des éléments.\\
%Scott Rogers Level Up!
%Game design foundations, R. Pedersen
%Game design theory & practice Second Edition, Rouse Richard
%A Systematic Review of Game Design Methods and Tools, Gomez, Jaccheri, Hauge
%GAMASUTRA Game Design Methods: A 2003 Survey, Bernd Kreimeier 



\section{Le concept document}
\subsection{ : Le <<\emph{one-page}>> de Librande \cite{onepage_librande}}

\GT{<<Le concept document>> ou <<La documentation des concepts du jeu>>?}
\EH{C'est bien le concept document, le one page est un document concis décrivant le concept général d'un jeu, on n'entre pas dans les détails on ne passe que sur les lignes directrices}



%One page designs, Stone Librante (ppt presentation)
%Game design foundations, R. Pedersen
%Scott Rogers Level Up!
Librande~\cite{onepage_librande} estime que plus un document est long, moins les utilisateurs s'y réfèrent.
Afin de répondre à cette problématique il développe le \emph{One-Page Design}.\\
Un document \emph{One page design} est une vue d'ensemble du jeu.
Il est destiné à l'équipe de développement et aux acteurs décisionnels des studios.
Il doit donc contenir les informations essentielles pour présenter le projet de jeu sous forme visuelle sur une seule page~\cite{LevelUpRogers2014}.

\gt{Le premier paragraphe et le deuxi\`eme contiennent des \'el\'ements r\'ep\'etitifs, non?}
\eh{J'ai regroupé les deux paragraphes et enlevé la répétition}
\gt{A quoi r\'ef\`ere <<ces \'el\'ements>? Pas clair!?}



\gt{Pas clair que c'est une autre proposition --- Librande vs.\
Rogers!? A clarifier/expliciter!}


\gt{Evite d'utiliser les items de gls directement dans une phrase, car pas clair.  Apr\`es un long tiret ou entre parenth\`ese semble pr\'ef\'erable.}

\gt{Et pas n\'ecessaire d'utiliser partout, car cela alourdit le texte
--- car cela met l'item accentu\'e, ce qui ne me semble pas
appropri\'e. }
\eh{J'ai modifié les *gls pour ne les afficher que la première fois qu'ils apparaissent et je les met tous entre --- aftin d'alléger leur apparition}


Voici les éléments essentiels d'un \emph{One-page design} tel que d\'ecrit par Librande~\cite{onepage_librande}:
\begin{itemize}
    \item Un titre représentatif du design
    \item Les dates des divers éléments, et ce afin de garder un historique
    \item Des espaces entre les informations, pour ne pas créer des blocs d'informations
    \item Une illustration centrale pour focaliser l'attention
    \item Sous l'illustration, possiblement une description et des textes explicatifs
    \item Pour des détails supplémentaires, des légendes autour de l'illustration ---
     ces légendes peuvent être des illustrations, elles-mêmes accompagnées de notes
    \item Des barres latérales pour ajouter des \emph{checklists}, des objectifs principaux ou des informations diverses
\end{itemize}

Selon Librande, la taille des éléments est importante dans un \emph{One-page design}: les tailles indiquent l'importance de l'information.
Si nécessaire, un \emph{One-Page design} peut être étendu à la taille nécessaire pour contenir toute l'information, y compris sous forme d'un poster.
Cependant, l'augmentation de taille de la page doit quand même permettre d'assurer la lisibilité --- augmenter en taille ne doit pas conduire à intégrer trop d'informations, ce qui irait à l'encontre de la lisibilité.


\subsection{Le <<\emph{one-sheet}>> de Rogers \cite{LevelUpRogers2014}}

Dans son article Rogers présente le <<\emph{one-sheet}>> comme vue d'ensemble du jeu. Il insiste sur le fait que ce document doit être intéressant, informatif et court.
Voici les éléments que doit contenir le \emph{One-Sheet} de Rogers~\cite{LevelUpRogers2014}.
\begin{itemize}
    \item Le titre du jeu
    \item Les systèmes de jeu prévus
    \item L'age des joueurs visés
    \item La notation ESRB --- voir plus bas
    \item Un résumé de l'histoire du jeu en se concentrant sur le \emph{gameplay}
    \item Les modes du \emph{gameplay}
    \item Les USP --- voir plus bas
    \item Les jeux compétiteurs
\end{itemize}

\gt{Jeux et logiciels? Ou juste jeux?}

Le ESRB --- \gls{esrb} --- est un organisme d'autorégulation qui est à l'origine d'un système de notation et de règles de respect de la vie privée des jeux et logiciels.
Son système de notation permet de définir à partir de points précis l'âge conseillé pour l'utilisation d'un jeu ou logiciel.
C'est une notation que l'on retrouve systématiquement dans le domaine du jeu vidéo définissant le type de contenu présent dans le jeu et le public pour lequel est recommandé ce contenu : eC (\emph{Early Childhood}), E (\emph{Everyone}), E10 (\emph{Everyone~$10^+$}), T (\emph{Teen}), M (\emph{Mature} $17^+$), AO (\emph{Adults Only} $18^+$).


\gt{Est-ce vraiment des <<points>> au sens de <<pointage>>
(num\'eriques)? Ou juste des <<arguments>> --- autre sens du terme
<<point>> en anglais~: selling point = argument de vente!?}

Les USP --- \gls{usp} --- sont des arguments de vente représentant le jeu.
Ils se situent à l'arrière de la boite de jeu ou sur le descriptif du jeu dans le cas d'une vente dématérialisée.
Ce sont des points précis et courts attirant la curiosité de l'acheteur sans être trop descriptifs. 


\gt{Je ne comprends pas le d\'ebut de la phrase suivante --- produits comp\'etitifs? Comp\'etiteurs?}

Les jeux compétiteurs liste les concurrents actuels du jeu, c'est-\`a-dire, les jeux déjà présents sur le marché.
Cette liste doit contenir des jeux connus ou connaissant un grand succès, afin qu'ils soient représentatifs du type du jeu d\'ecrit par le \emph{one-sheet}.
Cela permet donc de donner une bonne idée de ce que le jeu sera. 
Présenter une liste de jeux obscurs ou qui n'ont connus que peu de succès découragera les éditeurs qui liront le \emph{one-sheet}.

\gt{Ci-bas: Comme c'est une liste d'\'elements, il faut que tous
soient des substantifs --- et non des actions (verbes).}






\section{Étendre le \emph{One-page} vers un résumé}
Une fois les premières étapes de \emph{Game design} effectuées, le document \emph{One-page} va être étendu afin de produire un document plus important.
Ce second document va préciser les concepts du jeu vidéo en cours de création et va permettre de communiquer un niveau de détails plus précis.

%Game design foundations, R. Pedersen
Pedersen propose le \emph{five pager}~\cite{GD_foundations_pedersen}. C'est un résumé du concept du jeu vidéo et une description du jeu à venir.
Il comprend toutes les informations essentielles au cycle de vie du jeu sous ses aspects majeurs, tels que le \emph{gameplay}, l'audience cible, le scénario, et les \emph{features}.
Le \emph{five-pager} permet de présenter le jeu de manière plus poussée que e \emph{One-Page} à un décideur ou à un éditeur.

\gt{Ci-haut: cela me semble contradictoire/ambigue/m\'elangeant de
dire que le document de 5 pages pr\'esente le jeu sous sa forme <<la
plus basique>>, alors qu'on a au pr\'elable produit un document d'une
seule page. Reformuler?}


\gt{Il faut \'eviter de mettre dans le corps du texte des longues
s\'eries de d\'etails tel que le contenu du \emph{Ten-pager}.  Dans le
cas pr\'esent, je crois qu'une table serait plus appropri\'ee.}


\gt{Pour une telle liste/table, pr\'ef\'erable de ne pas mettre
d'article, donc <<Titre du jeu>> plut\^ot que <<Le titre du jeu>>.  De
plus, comme \'evoqu\'e pr\'ec\'edemment, il faut \^etre uniforme,
i.e., substantifs partout --- et non pas m\'elanger substantifs et
verbes (actions).}


Quant \`a Rogers~\cite{LevelUpRogers2014}, il  propose un document plus \emph{fourni} avec son \emph{Ten-pager}.
Le contenu du document doit non seulement contenir des informations pour l'équipe de développement du jeu, mais également les informations nécessaires pour int\'eresser les éditeurs impliqués dans le projet.
Rogers présente le contenu du \emph{Ten-pager} sous la forme suivante indiqu\'ee dans le tableau~\ref{ten-pager.table}.
%
\ACOMPLETER{Alors que les pages X \`a Y sont destin\'ees plus
sp\'ecifiquement \`a Z, les pages U \`a V sont destin\'ees \`a W.}
    
\gt{Paragraphe ci-haut~:  Tu parles de deux parties avec destinataires distincts. Ce serait bien d'expliciter}

\begin{table}[H]
\footnotesize
\begin{framed}
\begin{itemize}
    \item Page 1~: Informations générales
    \begin{itemize}
        \item Titre du jeu
        \item Systèmes de jeu prévus
        \item \^Age des joueurs visés
        \item Notation ESRB
        \item Calendrier prévisionnel de sortie
    \end{itemize}
    \item Page 2: Histoire
    \begin{itemize}
        \item  Résumé de l'histoire du jeu permettant de poser les premiers jalons de manière succinte et générale
        \item Résumé du déroulement du jeu  permettant de situer les actions du joueur dans l'histoire, les d\'efis rencontrés par celui-ci, comment se déroule la progression, la place du \emph{gameplay} dans l'histoire, les conditions de victoire, etc.
    \end{itemize}
    \item Page 3 : Détails du personnage contr\^ole par le joueur
    \begin{itemize}
        \item Histoire du personnage --- caractère, traits importants de son apparence
        \item \emph{Gameplay} particulier associé au personnage, ses mouvements, ses armes
        \item Maquette des contrôles proposés au joueur, par ex., représentation des raccourcis claviers
    \end{itemize}
    \item Page 4~: \emph{Gameplay}
    \begin{itemize}
        \item Modèle de séparation de l'histoire (niveaux, chapitres, monde ouvert)
        \item Scénarios particuliers (cinématique active)
        \item Mise en avant des USP
        \item Diagrammes et illustrations apportant des précisions
    \end{itemize}
    \item Page 5
    \begin{itemize}
        \item Images et description du monde du jeu
        \item Découpage des zones de jeu
        \item Liens entre les zones
    \end{itemize}

\end{itemize}
\end{framed}
\caption{Contenu du \emph{Ten-pager} selon Rogers~\cite{LevelUpRogers2014}.}
\end{table}

\addtocounter{table}{-1}

\begin{table}[H]
\footnotesize
\begin{framed}
\begin{itemize}
    \item Page 6~: Expérience de jeu
    \begin{itemize}
        \item Description des émotions et sensations que doit générer l'expérience de jeu
        \item Description de l'interface de jeu et de la manière d'en faire usage

\gt{A quoi r\'ef\`ere <<les>> dans <<les parcourir>>? Pas clair!}

    \end{itemize}
    \item Page 7~:  Mécaniques de \emph{gameplay}
    \begin{itemize}
        \item Mécaniques : Eléments avec lequel un joueur interagit pour effectuer des actions (par ex., levier pour ouvrir une porte)
        \item Dangers : Eléments du monde pouvant tuer ou blesser le joueur mais sans IA (par ex., bloc de pierre qui tombe)
        \item \emph{Power-up} : Objets que le joueur récupère et utilise pour obtenir un avantage (par ex., champignon dans Mario)
        \item Objets de collections : Objets que le joueur collecte mais qui n'influence par directement le \emph{gameplay} (par ex.,  monnaie)
    \end{itemize}
    \item Page 8~: Ennemis
    \begin{itemize}
        \item Description des ennemis rencontrés par le joueur et qui sont contrôlés par une IA
        \item Description des \emph{boss} rencontrés
    \end{itemize}
    \item Page 9~: Multijoueur et bonus
    \begin{itemize}
        \item Description des succès collectionnables
        \item Description des secrets découvrables
        \item Description des interactions si le jeu est multijoueur
    \end{itemize}
    \item Page 10: Monétisation
    \begin{itemize}
        \item Description du système de monétisation du jeu (Gratuit, Gratuit avec une boutique en jeu, payant à l'achat, etc.)
        \item Description des boutiques et de leur contenu
    \end{itemize}
\end{itemize}
\end{framed}
\caption{Contenu du \emph{Ten-pager} selon Rogers~\cite{LevelUpRogers2014} (suite).}
\label{ten-pager.table}
\end{table}




\section{Le \emph{Game Design Document}}
%Game design foundations, R. Pedersen
%Proposal of Game Design Document from Software Engineering Requirements Perspective, Mario Gonzalez Salazar, Hugo A. Mitre, Cuauhtémoc Lemus Olalde, José Luis González Sánchez

%Incorporating a Game Design Document into Game Development Projects Deliverables,Craig Marais,Lynn Futcher,Johan van Niekerk
%GAMASUTRA Creating a great design document Tzvi Freeman

%Improving Game Development Process Applying Multi-View Game Design Documents,Pablo Correa , Luisa Möller 
%6


%GAMASUTRA 5 Alternatives to a Game Design Document, Erin Robinson

\subsection{Le GDD, un moyen de communication}
Le \gls{gdd} est un des documents les plus complet trouvable dans le cadre du développement d'un jeu vidéo.
Dans la littérature il est souvent qualifié de <<Bible du design>>~\cite{GD_foundations_pedersen} --- bien que certains auteurs ne soient pas d'accord sur ce sujet~\cite{LevelUpRogers2014}. 

Le GDD doit contenir, dans l'idéal, toute la vision du \emph{game designer} sans laisser de doute sur ce que celui-ci veut représenter et souhaite voir dans le jeu.
Peu importe le corps de métier de la personne lisant le GDD, celle-ci doit pouvoir se représenter au maximum le rendu final attendu.
Le GDD doit donc pouvoir servir de moyen communication entre les différentes équipes entourant le développement du jeu, et ce \`a toutes les étapes, depuis la création de l'idée jusqu'au post-mortem du projet.

\subsection{Une structure souple mais un contenu précis}

\gt{Lorsqu'on indique un auteur, il suffit de mettre le nom de famille, sans le pr\'enom.}

Dans un article sur Gamasutra~\cite{gama_greateGDD}, Freeman établit une liste de 10 pratiques qu'il estime nécessaire de respecter pour d\'ecrire un GDD :
\begin{itemize}
    \item Décrire le corps du jeu, mais également son âme
    \item Rendre le GDD lisible
    \item Etablir des priorité dans les tâches
    \item Rentrer au maximum dans les détails
    \item Démontrer et décrire les sujets
    \item Décrire le <<quoi>> mais également le <<comment>>
    \item Proposer des alternatives de mises en oeuvres \gt{Pas clair: des alternatives de mises en oeuvres?}
    \item Donner une vie au GDD
    \item Pr\'eciser la vision du \emph{Game designer} de fa\c{c}on suffisamment précise pour ne pas laisser d'éléments non décrits
    \item Mettre à disposition le GDD dans de bonnes conditions
\end{itemize}

\subsection{Un GDD complet mais pas surchargé}

\gt{Ci-bas~: j'ai reformul\'e la premi\`ere phrase car c'\'etait vague
de parler <<d'outils>> et c'\'etait bizarre de m\'elanger outils et
libert\'e: A revoir!}

Le GDD doit \^etre un outil pour les \emph{game designers}, tout en leur donnant la liberté de créer de nouveaux jeux, de nouveaux \emph{gameplay}, de nouveaux concepts, donc sans les brider dans leur cr\'eativit\'e.
De nombreux travaux essaient de répondre au besoin de formalisation du GDD~\cite{GDD_software,multiview,GDD_GDProject,gama_greateGDD} sans pour autant réussir à apporter une solution permettant d'englober tous les types de jeux, tous les types de métiers ou toutes les étapes de développement.
Et de nombreux problèmes annexes se posent concernant l'universalité d'un modèle de GDD : Est-il complet ? Est-il assez précis ? Est-il assez souple ? Est-il utile à l'équipe ? 

Prenons l'exemple d'un projet où le GDD est extrêmement complet, comportant tous les détails voulus par le \emph{Game designer}.
Le GDD peut alors devenir tellement complet qu'il en devient imposant, lourd à parcourir, difficile à maintenir et à modifier.
Il perdra alors toute l'efficacité de design qu'il aurait pu donner, pouvant devenir un handicap de temps plutôt qu'une aide au développement~\cite{onepage_librande}.

Un GDD est l'atout majeur de la phase de production d'un jeu vidéo et pour qu'il soit utile durant toutes les phases celui-ci doit être impérativement complet et sans ambiguité~\cite{GD_Guidelines}.
Cependant il doit être assez souple afin de suivre le jeu dans son développement et suivre les changements nombreux et soudains qui peuvent arriver au cours du développement.

Dans son article~\cite{gama_greateGDD}, Freeman écrit que le GDD doit non seulement décrire le corps du jeu vidéo mais son <<âme>>.
Construire un GDD selon lui doit être un moyen de représenter le résultat attendu dans les moindres détails pour permettre aux équipes de travailler sur une idée précise.
Faire appel à des outils graphiques plutôt qu'à des textes peut être un moyen d'apporter plus de précision à une idée.
Mais le GDD doit être un moyen d'expliquer toutes les parties du jeu, le <<quoi>>, mais également le <<comment>>.

Cependant décrire l'intégralité des détails dans le GDD peut mener à des dérives décrites par Rouse~\cite{GD_theory_rouse}.
À vouloir apporter trop de détails dans un GDD, un \emph{Game designer} peut perdre énormément de temps qui aurait pu être économisé avec plus de communication avec son équipe.
Trop de détails peut également apporter trop de limitations aux corps de métier plus artistiques et brider leur créativité.
Certaines parties du jeu peuvent également en obstruer d'autres : trop de détails dans la description d'une mécanique peut éclipser le fait qu'une autre n'est pas assez détaillée.
Un GDD trop complet donne également l'impression que le projet est terminé et qu'il ne nécessite plus d'ajustements, cela nuit à l'évolution du jeu et aux modifications positives qui pourraient lui être appliquées.


\GT{Il va falloir que je relise cette derni\`ere section --- en fait,
le chapitre au complet --- car j'ai l'impression qu'il y a certaines
redondances/r\'ep\'etitions, mais aussi quelques aspects manquants
dans le fil logique.}



\section{Conclusion}
Dans ce chapitre, nous avons vu trois types de documents de \emph{Game design}~: 
\begin{itemize}
    \item Le \emph{One-Page} et le \emph{One-Sheet} qui servent de premier aperçu du jeu
    \item Le \emph{Five-Pager} et le \emph{Ten-Pager} qui permettent de présenter le jeu à une équipe de développement et permettent de référencer les premières pistes de \emph{Game design}
    \item Le \emph{Game Design Document} qui permet de regrouper tous les éléments de design pour un jeu en cours de conception ou de développement
\end{itemize}



\gt{Les nommer ici, comme rappel/r\'esum\'e/synth\`ese}
%  
Ces documents sont les plus répandus dans le monde du développement de jeux vidéos. 

Selon le type de \emph{gameplay}, la quantité d'informations ou la durée de vie du jeu, un GDD peut avoir une taille plus ou moins importante. 
%
La structure du GDD peut différer d'un \emph{Game designer} à l'autre en fonction des habitudes de travail avec son équipe ou son expérience.
Le contenu du GDD peut aussi être sous formes diverses :  textes,  graphiques, diagrammes,  \emph{mind mapps}. 

%La structure de tels documents n'est pas fixe et leur rédaction peut \^etre effectuée en suivant des bonnes pratiques décrites par des \emph{Game designer} expérimentés et trouvent réellement leur forme par la manière dont un \emph{Game designer} souhaite partager sa vision du futur jeu.

\gt{Derni\`ere phrase \`a relire!!!}
\eh{Je l'ai modifiée et reformulée pour celle en dessous.}

C'est pour ces raisons qu'il est compliqué d'établir un gabarit précis pour la rédaction d'un GDD.
Aucun standard n'existe afin de définir précisément l'architecture des documents de \emph{game design}.
La rédaction de ces documents se fait en suivant des bonnes pratiques décritent par des \emph{game designers} expérimentés.
Chaque \emph{game designer} peut définir son propre template en fonction de ses besoins et de ses habitudes.
\chapter{Le framework Mechanics, Dynamics, Aesthetics}

La conception de jeu est le processus de réflexion nécessaire à la création d'un jeu vidéo ou de n'importe quelle sorte de jeu.
C'est le chemin que suit le jeu depuis une idée jusqu'à sa réalisation.
De nombreuses étapes permettent de mettre en place tous les aspects du jeu pour le mener à sa version jouable.
La mise en place du monde dans lequel le joueur évoluera, les règles qu'il devra suivre pour avancer, les éléments graphiques qu'il rencontrera tout cela est défini lors de la conception. 

Le jeu vidéo est cependant difficile à concevoir et surtout à documenter.
Un logiciel classique répond au besoin d'un utilisateur qui s'en sert pour résoudre une problématique.
Un jeu quant à lui doit apporter un but au joueur, l'envie de continuer à l'utiliser et de générer des émotions pour être apprécié.
Il est très compliqué de définir une méthodologie précise pour décrire une émotion générée ainsi que tous les éléments techniques autour d'un jeu vidéo. 
C'est ce que le Framework \gls{mda} propose.


\section{Le jeu : un échange entre \emph{Game Designer} et Joueur}
Un \emph{Game Designer} créé un jeu dans le but de générer une expérience de jeu. 
Cependant l'utilisation que le joueur fera du produit fini est imprévisible \cite{MDA_formal}. 
Dans leur article Hunicke, LeBlanc et Zubek \cite{MDA_formal} découpent un jeu en trois aspects distincts : les règles, le système et le \guillemotleft fun \guillemotright. 
Ces trois aspects sont à l'origine des trois parties du Framework MDA comme indiqués dans la Figure \ref{fig:mda}.
\begin{figure}[H]
    \begin{center}
    \includegraphics[width=14cm]{10_img/chap3/mda.png} 
    \caption{Contenu des parties du Framework MDA \cite{MDA_formal}}
    \label{fig:mda}
    \end{center}
\end{figure}


%MDA: A Formal Approach to Game Design and Game Research Robin Hunicke, Marc LeBlanc, Robert Zubek
%A better recipe for game jams: using the Mechanics Dynamics Aesthetics framework for planning Paris Buttfield-Addison,Jon Manning,Tim Nugent
%\Design, Dynamics, Experience (DDE): An Advancement of the MDA Framework for Game Design
\section{\emph{Mechanics}}

\label{mechanics.sect}

Les \emph{Mechanics} du jeu sont tous les éléments composant le jeu au niveau de la représentation des données et des algorithmes. 
Les éléments sont dans des catégories qui permettent non seulement de les classer mais également d'aider à les définir plus précisément. 
Ces catégories peuvent comprendre des attributs et des spécificités qui sont appliquées aux éléments contenus dans les catégories.

Les \emph{Mechanics} comprennent également les actions, comportements et mécanismes de contrôles mis à disposition du joueur. 
Cela peut correspondre aux mouvements d'un personnages, aux actions possibles sur des objets ou aux interactions entre les objets. Les règles sont aussi définies dans les mécaniques.


\section{\emph{Dynamics}}

Les \emph{Dynamics} sont les conséquences des \emph{Mechanics}. 
Elles décrivent le comportement de l'exécution des \emph{Mechanics} lorsqu'elles seront utilisées par le joueur \cite{GAMA_MDA}. 
Ces \emph{Dynamics} sont importantes à prévoir car c'est elles qui permettront l'évolution du joueur dans le jeu.

Afin de définir les \emph{Dynamics} il est possible d'analyser d'autres jeux (de même genre ou opus précédents) mais il est aussi possible d'effectuer des calculs statistiques sur des habitudes de jeu, étudier la psychologie des joueurs afin de prévoir leur \emph{gameplay} en jouant sur les émotions.


\section{\emph{Aesthetics}}
Les \emph{Aesthetics} sont, selon Hunicke, LeBlanc et Zubek \cite{MDA_formal}, \guillemotleft ce qui rend le jeu \emph{fun} \guillemotright . 
Ce sont toutes les émotions générées par le jeu et transmises au joueur via des mécaniques de jeu, les sons ou les graphismes. Ils classifient ces mécanismes en huit catégories :
\begin{itemize}
    \item Sensation : le jeu comme plaisir des sens
    \item Fantaisie : le jeu comme imaginaire
    \item Narration : le jeu comme drama
    \item Challenge : le jeu comme un parcours d'obstacles
    \item Communauté : le jeu comme un réseau social
    \item Découverte : le jeu comme un territoire inexploré
    \item Expression : le jeu comme une découverte de soi-même
    \item Soumission : le jeu comme un passe temps
\end{itemize}

Le jeu final peut contenir une ou plusieurs catégories de \guillemotleft fun \guillemotright et l'ensemble de l'expérience du joueur repose sur ces \emph{Aesthetics} . 
Cette dernière partie du \emph{Game design} est tellement importante qu'il est possible de concevoir et développer le jeu en ayant au préalable sélectionné un certain nombre de ces catégories et en les considérant comme objectifs à atteindre.




\section{Les limitations et évolutions possibles}
\subsection{DPE : Le Framework Design, Play, Experience}
%ARTCILE The Design, Play, and Experience Framework, Brian M. Winn

\begin{figure}[H]
    \centering
    \includegraphics[width=10cm]{10_img/chap3/dpe.png} 
    \caption{Contenu des parties du Framework DPE \cite{Winn2011}}
    \label{fig.dpe}
\end{figure}

Le \gls{dpe} présenté dans la figure \ref{fig.dpe} est un framework reposant sur les mêmes principes que le MDA . 
Des modifications sont appliquées au MDA afin d'étendre ses capacités de design pour les jeux sérieux. 
Il étend le framework MDA afin d'y intégrer plus facilement les notions : d'apprentissage, de  \emph{story telling}, de  \emph{gameplay} et de composants technologiques spécifiques aux jeux sérieux. 
Dans la figure \ref{fig.dpe_extended} sont présentés les différentes parties du DPE présentées par Winn \cite{Winn2011}.


\begin{figure}[H]
    \centering
    \includegraphics[width=14cm]{10_img/chap3/dpe_extended.png} 
    \caption{Framework DPE étendu \cite{Winn2011}}
    \label{fig.dpe_extended}
\end{figure}


Dans le DPE le \emph{Game Designer} a le contrôle direct sur l'ensemble des catégories. 
Afin d'avoir le contrôle sur la partie Expérience le \emph{Game Designer} définit des objectifs d'\emph{Experience}. 
Dans la figure \ref{fig.dpe_iteratif} c'est ce qui est représenté par la flèche entre Experience et Design. 
Cette flèche représente également le processus itératif du design décrit par Salen et Zimmermann \cite{Salen2013}. 
La conception permet de produire un prototype, et l'expérience sur ce prototype permet de modifier le design afin que l'Experience corresponde aux attentes du \emph{game designer}.

\begin{figure}[H]
    \centering
    \includegraphics[width=8cm]{10_img/chap3/iteration_prototype.png} 
    \caption{Processus de design itératif \cite{Winn2011}}
    \label{fig.dpe_iteratif}
\end{figure}





\subsection{DDE : Le framework Design, Dynamics, Experience \cite{DDE}}
%ARTICLE DDE Design, Dynamics, Experience (DDE): An Advancement of the MDA framework for Game Design Wolfgang

%GAMASUTRA From MDA to DDE, Wolfgang Walk
Le framework \gls{dde} décrit dans la figure \ref{fig.dde} est présenté dans l'article de Walk \emph{et al.} \cite{DDE}. 
Ils apportent une vision critique sur le framework MDA et avancent que celui-ci néglige beaucoup d'aspects du design de jeux vidéos. 
Ils estiment également que le MDA se concentre trop sur les \emph{Mechanics} et ne permet donc pas de décrire tous les types de \emph{gameplay} présents sur le marché du jeu vidéo. 
Ce focus sur les \emph{Mechanics} entraîne, selon Walk \emph{et al}, un manque de contrôle du \emph{game designer} sur les \emph{Dynamics} et \emph{Aesthetics} qui ne font que découler des \emph{Mechanics}

\begin{figure}[H]
    \begin{center}
    \includegraphics[width=10cm]{10_img/chap3/dde.png} 
    \caption{Contenu du Framework DDE \cite{DDE}}
    \label{fig.dde}
    \end{center}
\end{figure}

Afin de définir le framework DDE ils effectuent un découpage différent et intègrent de nouvelles notions dans les catégories comme décrit dans la figure \ref{fig.dde_extended}.

\begin{figure}[H]
    \begin{center}
    \includegraphics[width=13cm]{10_img/chap3/dde_extended_modif.png} 
    \caption{Framework DDE étendu \cite{DDE}}
    \label{fig.dde_extended}
    \end{center}
\end{figure}

\subsubsection{Design}
    \begin{itemize}
        \item \emph{Blueprint} : Partie du design qui concerne les concepts du monde du jeu : culture, religion, physique, les différents sets de règles, les styles artistiques, le design narratif, le design de personnages et le design sonore qui ensemble créent l'expérience esthétique.
        \item \emph{Mechanics} : Toute chose créant le jeu, plus précisément le code. L'architecture du code, la prise en charge des entrées/sorties, la prise en charge des objets, l'implémentation des règles de jeu et l'interaction entre les objets, et tous les éléments reliés au code. Cela comprend tous les éléments que le joueur ne perçoit pas dans son utilisation du jeu.
        \item Interface : Toutes les mécaniques qui ont pour but de communiquer le jeu au joueur. Les graphismes, le son, les réactions et interactions entre le joueur et le jeu, ainsi que celles bouclant sur le jeu lui-même. Cette partie comprend également les cinématiques les textes affichés et tout ce qu'il est possible de voir ou d'entendre dans le jeu.
    \end{itemize}

\subsubsection{\emph{Dynamics} }
    La catégorie \emph{Dynamics} du DDE correspond à celle présente dans le MDA, mais elle classifie les interactions de manière à les rendre plus précises. Cest ainsi que ces Dynamics se retrouvent en trois grandes catégories: 
    \begin{itemize}
        \item Player <-> Game
        \item Player <-> Player
        \item Game <-> Game
    \end{itemize}

\subsubsection{\emph{Experience}}
    Dans le MDA la troisième partie du framework est l'\emph{Aesthetics} : tout ce que le joueur sent et ressent lors de son activité sur le jeu. 
    La partie Expérience du jeu étend l'\emph{Aesthetics} afin de prendre en considération que le joueur n'est pas une somme des émotions générées par le jeu. 
    Le joueur devient un élément avec une expérience déjà présente avant l'utilisation du jeu ce qui peut modifier les émotions générées d'un joueur à l'autre. Une même couleur, un même son ou une même image peuvent générer différentes réactions de la part du joueur et la partie Expérience du DDE essaie de prendre en compte cela.
    \begin{itemize}
        \item Senses: expérience sensorielle du joueur du début à la fin du jeu.
        \item Cerebellum : les émotions ressenties par le joueur.
        \item Cerebrum : les challenges intellectuels et les décisions prises par le joueur.
        \item Player<Subject> : partie que le designer ne peut pas contrôler. Il peut se baser sur les trois premières catégories afin de prévoir une réponse spécifique du joueur. N'ayant aucun contrôle sur celle-ci il ne peut qu'estimer la réaction du joueur en fonction d'objectifs et de logiques psychologiques pour l'amener à la réaction souhaitée.
        \item Perception : ce que ressent réellement le joueur en fonction des trois premières catégories et de sa propre expérience et personnalité en tant que joueur. Cela comprendra son \emph{gameplay}, le type de challenge qu'il perçoit, l'amusement qu'il ressent, la beauté qu'il perçoit, l'écho que génère l'histoire en lui, etc.
    \end{itemize}


%GAMASUTRA Revisiting the MDA framework, Luiz Claudio Silveira Duarte
\subsection{Le MDA est limité à la représentation des jeux vidéos}
Dans son article Duarte \cite{GAMA_MDA} met en avant des difficultés que l'on rencontre à représenter des jeux qui ne soient pas des Jeux vidéos en faisant usage du MDA. 
Il explique que les règles d'un jeu vidéo sont souvent implicites au type de \emph{gameplay} ou de genre de classification de jeu vidéos. 
Cependant dans le cadre des jeux de plateau les règles ne peuvent pas être implicites et acquises par l'expérience. 
Elles doivent être explicites et explicables dans un manuel d'utilisation. 
Il fait l'analogie entre un \gls{fps} et un jeu d'échecs. Dans un FPS un joueur sait à quoi s'attendre selon son expérience du genre. 
Il saura ou trouver les éléments et saura comment faire usage de l'interface graphique qui lui est présentée. 
Cependant un joueur se retrouvant devant un jeu d'échec pour la première fois ne pourra pas acquérir les connaissances requises pour jouer par l'expérience, les règles devront lui être expliquées à la base et l'expérience ne pourra lui apporter que des aspects tactiques du jeu.

\section{Conclusion}

\chapter{Les profils UML}
%OMG Unified Modeling LanguageTM (OMG UML),Superstructure

\section{Le langage de modélisation UML}
\gls{uml} est un langage de modélisation standardisé permettant de créer des diagrammes.
Ces diagrammes permettent de visualiser, spécifier, construire et documenter des logiciels, des systèmes et des processus d'affaire.
UML utilise des notations graphiques pour exprimer le design et l'architecture de projets logiciels.
Les spécifications du standard UML sont en ligne sur le site de l'OMG (Object Management Group)  \cite{OMG_UML}.
Dans la figure \ref{fig.uml_struc} sont représentés les diagrammes de structure proposés par UML.
Dans la figure \ref{fig.uml_comp} sont proposés les diagrammes de comportement proposés par UML.

\begin{figure}[H]
    \centering
    \includegraphics[width=12cm]{10_img/chap4/structure.PNG} 
    \caption{Diagrammes de structure dans UML \cite{OMG_UML}}
    \label{fig.uml_struc}
\end{figure}

\begin{figure}[H]
    \centering
    \includegraphics[width=12cm]{10_img/chap4/comportement.PNG} 
    \caption{Diagrammes de comportement dans UML \cite{OMG_UML}}
    \label{fig.uml_comp}
\end{figure}

UML est un langage de modélisation connu et largement documenté dans le domaine de l'informatique.
Nous nous concentrerons plus particulièrement à définir les notions nécessaires à la compréhension des mécanismes et des éléments composant les profils UML.

\section{Le diagramme de profil UML}
Un profil UML est une forme de diagramme structurel décrit dans le standard UML. 
Il permet d'étendre les mécanismes d'UML afin d'adapter les diagrammes et leur contenu à un domaine (ex : domaines d'activité) ou une plateforme particulière (ex : .NET, J2EE). 

Les extensions qu'un profil appliquées au langage UML permettent d'ajouter des caractéristiques aux éléments standards d'UML. 
Elles ne permettent pas de retirer les caractéristiques des éléments pour ne pas aller à l'encontre de la sémantique standard d'UML.
Le profil se compose  de stéréotypes de tagged values et de contraintes qui s'appliquent aux éléments des modèles UML tels que les classes, les attributs, les opérations et les activités.


\subsection{stereotypes}
Les stéréotypes permettent d'appliquer des extensions aux métaclasse UML. 
Il permettent d'ajouter des termes spécifiques de vocabulaire à un diagramme. 
Un stéréotype s'applique à un objet UML permettant ainsi de le rendre spécifique à un domaine en y appliquant des propriétés spécifiques.

\subsection{extensions}
link entre metaclass et stereotype


\subsection{contraintes}
ex : un stereotype peut generaliser ou specialiser un autre stereotype


\subsection{tagged values}


\subsection{Image}


\subsection{reponse a la pb}


\chapter{Un profil UML pour aider à la rédaction de GDD: \emph{Game Genesis}}
\label{chap.game-genesis}
\label{game-genesis.sect}

%GDD
Un \emph{Game Design Document}, tel que décrit à la Section~\ref{sect.GDD}, permet de réunir toutes les informations de design nécessaires au développement d'un jeu vidéo. La structure du GDD est définie en respectant des bonnes pratiques et selon les besoins de design du jeu concerné.


\gt{Ci-bas: il faut \^etre plus sp\'ecifique pour l'aspect Mechanics
car si tu dis <<{\bf tous} les \'el\'ements du jeu>>, alors pourquoi
faudrait-il autre chose que cela!?}

%MDA
Le \emph{Framework Mechanics, Dynamics, Aesthetics}, décrit au Chapitre~\ref{chap.MDA}, permet de séparer les différents aspects du design d'un jeu vidéo. L'aspect \emph{Mechanics} permet de représenter les éléments du jeu rattachés aux données et aux algorithmes, l'aspect \emph{Dynamics} décrit le comportement des \'el\'ements de \emph{Mechanics}, alors que l'aspect \emph{Aesthetics} d\'ecrit les émotions découlant de la \emph{Dynamics} que le jeu génère chez le joueur.

%UML
Les profils UML, décrits au Chapitre~\ref{chap.profils-UML}, permettent d'étendre les concepts présents dans UML. Mettre en place un profil UML permet de faciliter la cr\'eation d'un modèle et de son contenu pour un domaine particulier. Un profil se compose de stéréotypes, de valeurs étiquetées et de contraintes, qui ensemble permettent de d\'efinir de nouveaux concepts utilisables dans la description de modèles.


\section{Un aper\c{c}u de \emph{Game Genesis}}
\label{sect.gg_what}
%quoi
\emph{Game Genesis} est un profil UML que nous avons d\'evelopp\'e et qui permet d'adapter UML au domaine du design de jeux vidéos. 
Plus précisément, \emph{Game Genesis} introduit de nouvelles classes et associations --- par le biais de st\'er\'eotypes --- afin de décrire les {\bf éléments de \emph{Mechanics}} d'un GDD.

\begin{comment}
\GT{J'ai mis ici la partie ci-bas (2 paragraphes), qui était
auparavant à la fin de la section, car sinon ça semblait
redondant. Par contre, je me demande si elle ne devrait pas simplement
être supprimé, car elle semble simplement répéter ce qui a été dit à
la fin du chapitre précédent, non? Il me semble que oui, donc je mets
en commentaire: à vérifier si tu es d'accord.}

%pourquoi
L'utilisation de diagrammes de classes UML apporte certains avantages pour la modélisation des \'el\'ements de \emph{Mechanics}:
\begin{itemize}
    \item Structure;
    \item Langage de modélisation formel et normalisé;
    \item Support de communication visuel; \gt{Pas trop certain de ce que cela veut dire, i.e., un peu vague: <<performant>>!}
    \item Facilement versionnable
    \item Assez souple pour s'adapter aux domaines ou aux plateformes de développement; \gt{s'adapter \`a quoi!?}
    \item Utilisation d'outils adaptés.
\end{itemize}

Cependant, UML est un langage avant tout destiné à modéliser les systèmes informatiques et les processus d'affaire. 
Il est donc nécessaire de l'adapter afin de lui apporter le vocabulaire et les logiques nécessaires à la description de l'aspect \emph{Mechanics} d'un jeu vidéo.
C'est à cela que sert le profil \emph{Game Genesis}.
   
\GT{Fin de la partie déplacée, qui devrait probablement être supprimée, non?}
\end{comment}

\gt{Ci-bas et ailleurs. Je ne crois pas que tu devrais utiliser
directement le terme Mechanics comme dans <<un certain nombre de
Mechanics>>.  Comme indiqu\'e pr\'ec\'edemment (vieille remarque GT),
il me semble que Mechanics, en anglais, est plus compris au sens de
<<la m\'ecanique>> --- m\'ecanique des fluides, m\'ecanique quantique,
etc.  C'est pour cela que, plus haut, j'ai chang\'e pour <<l'aspect
Mechanics>>, que ci-bas j'ai indiqu\'e <<\'el\'ements de Mechanics>>.}

\gt{Et je viens de regarder l'article initial sur MDA.  L\`a aussi,
les termes Mechanics, Dynamics et Aesthetics sont consid\'er\'es comme
des substantifs au singulier: <<Mechanics describes the particular
components of the game, ...>>, <<Dynamics describes the run-time
behavior...>>, <<Aesthetics describes...>>.}


\gt{Et le terme <<aspect>> me semble aussi appropri\'e --- bien que
celui de <<vue>> ou <<point de vue>> pourrait aussi l'\^etre: <<Each
component of the MDA framework can be thought of as a 'lens' or a
'view' of the game...>>.}



Afin de rédiger un GDD, un \emph{game designer} doit d'abord définir les bases du jeu.
Durant cette étape, le \emph{game designer} va devoir définir un certain nombre d'\'el\'ements de \emph{Mechanics} qui correspondent aux objets présents dans le jeu ainsi que leurs actions et interactions.
Ces \'el\'ements de \emph{Mechanics} peuvent aussi bien être des objets visibles dans le jeu --- comme des personnages, des ennemis, des armes, des bâtiments --- mais également des \'el\'ements autour du jeu --- comme des cartes, des chronomètres, des statistiques, etc.
Tous ces \'el\'ements de \emph{Mechanics} ont des caractéristiques qui leur sont propres.

\begin{figure}[H]
    \begin{adjustbox}{width=\linewidth}
        \begin{forest}
         [\texttt{GameGenesis}
         [\texttt{Item}
             [\texttt{Wearable},tier=2
                 [\texttt{Weapon},tier=before
                    [Fig.~\ref{A-Weapon},tier=bottom]
                 ]
                 [\texttt{Equipment},tier=before
                    [Fig.~\ref{A-Equipment},tier=bottom]
                 ]
                 [\texttt{Jewerly},tier=before
                    [Fig.~\ref{A-Jewerly},tier=bottom]
                 ]
                 [\texttt{Tool},tier=before
                    [Fig.~\ref{A-Tool},tier=bottom]
                 ]
             ]
             [\texttt{AddOn},tier=2
                    [Fig.~\ref{A-Add-on},tier=bottom]
             ]
             [\texttt{Usable},tier=2
                    [Fig.~\ref{A-Usable},tier=bottom]
             ]
             [\texttt{Craft},tier=2
                    [Fig.~\ref{A-Craft},tier=bottom]
            ]
             [\texttt{Currency},tier=2
                    [Fig.~\ref{A-Currency},tier=bottom]
            ]
         ]
         ]
        \end{forest}
    \end{adjustbox}
    \caption{L'arbre des stéréotypes de \emph{Game Genesis}.}
    \label{fig.GG}
    \label{fig.GG1}
\end{figure}
    
\begin{figure}[H]
    \begin{adjustbox}{width=\linewidth}
        \begin{forest}
         [\texttt{GameGenesis}
         [\texttt{Animate} ,tier=2
                [Fig.\ref{A-Animate},tier=bottom]
         ]
         [\texttt{CharacterSheet},tier=2
                [\texttt{Statistic},tier=before
                    [Fig.~\ref{A-Statistic},tier=bottom]
                ]
                [\texttt{Attribute},tier=before
                    [Fig.~\ref{A-Attribute},tier=bottom]
                ]
                [\texttt{Information},tier=before
                    [Fig.~\ref{A-Information},tier=bottom]
                ]
                [\texttt{Experience},tier=before]
                [\texttt{Ability},tier=before]
                [\texttt{Ranking},tier=before]
         ]
         [\texttt{Lore} ,tier=2
                [Fig.~\ref{A-Lore},tier=bottom]
         ]
         [\texttt{World} ,tier=2
                [Fig.~\ref{A-World},tier=bottom]
         ]
         [\texttt{Interaction},tier=2
                [Fig.~\ref{A-Interaction},tier=bottom]
         ]
         ]
        \end{forest}
    \end{adjustbox}
    \caption{L'arbre des stéréotypes de \emph{Game Genesis} (suite).}
    \label{fig.GG2}
\end{figure}

%comment
\emph{Game Genesis} permet à un \emph{game designer} de répertorier et décrire ces \'el\'ements de \emph{Mechanics} sous forme de diagrammes de classes.
Chaque catégorie de \emph{Mechanics} est représentée par une classe, et
chaque \'el\'ement de \emph{Mechanics} est aussi représenté par une classe.
L'appartenance à une catégorie est définie par des relations d'héritage.
Les caractéristiques des catégories et des divers \'el\'ements de \emph{Mechanics} sont représentées par des attributs.
Finalement, certaines interactions entre \'el\'ements du jeu peuvent être représentées par des associations entre classes, associations auxquelles peuvent être associ\'ees certaines contraintes.

Les figures~\ref{fig.GG} et~\ref{fig.GG2} présentent une vue d'ensemble des différents stéréotypes introduits par \emph{Game Genesis} --- seuls les niveaux supérieurs de la hiérarchie des stéréotypes sont indiqués.



\goodbreak

\section{\emph{Game Genesis} en détail}

\begin{samepage}
Dans les sous-sections qui suivent, nous définissons plus en détail les éléments suivants de \emph{Game Genesis}~: 
\begin{itemize}
    \item Les stéréotypes qui permettent d'étendre les classes.
    \item Les stéréotypes qui permettent d'étendre les associations.
    \item Les contraintes qui d\'efinissent des règles d'utilisation des stéréotypes.
\end{itemize}
\end{samepage}

\subsection{Les stéréotypes}

\gt{Dans la figure, comme ce sont des \'el\'ements de <<code>> ---
classes UML --- il vaut mieux utiliser la police teletype.}

\gt{Ce serait pr\'ef\'erable que dans les deux parties de la figure,
les r\'ef\'erences aux annexes soient pr\'esent\'ees de la m\^eme
fa\c{c}on pouar Animate, Lore, World et Interaction. Sinon, cela donne
l'impression que la s\'emantique pourrait \^etre diff\'erente ---
parce que c'est bien la m\^eme, n'est-ce pas?}


\begin{figure}
    \centering
    \includegraphics[width=\linewidth]{10_img/chap5/metaclass_class.PNG} 
    \caption{La spécification de certains stéréotypes de \emph{Game Genesis}, obtenus par \emph{extension} de la m\'etaclasse \texttt{Class}.}
    \label{fig.meta_class}
\end{figure}

\gt{Ci-bas: dans le cas pr\'esent, ce n'est pas tant une
<<red\'efinition>> qu'une extension, qu'une cr\'eation de nouvelles
classes, de nouveaux concepts! Donc, j'ai modifi\'e en cons\'equence.}

Dans la Section~\ref{sect.uml.ster}, nous avons décrit le mécanisme de d'extension de m\'etaclasses afin d'introduire de nouveaux concepts par le biais de stéréotypes. 
Dans \emph{Game Genesis}, nous faisons donc usage des stéréotypes afin d'étendre les classes d'un modèle de jeu et ainsi permettre la rédaction de GDD.
%
Les racines des arbres définissant ces divers stéréotypes sont
présentées aux figures~\ref{fig.GG} et~\ref{fig.GG2}, alors que les
détails des sous-arbres sont présentés en annexe, dans une figure
dont le numéro est indiqué sur ces figures.

La figure~\ref{fig.meta_class} présente les différentes <<~racines~>> de \emph{Game Genesis}. 
Ce sont les premiers stéréotypes du profil qui serviront majoritairement de catégories de haut niveau pour classifier les éléments de \emph{Mechanics} décrits dans un \emph{Game Design Document}.
Nous présentons le profil plus en détail dans l'Annexe~\ref{AnnexeA}.
%
Comme on le remarque, ces stéréotypes sont des \emph{extensions} de la
métaclasse \texttt{Class}, donc ils serviront à annoter des classes, des concepts.


Dans un article de Salazar \emph{et al.}~\cite{GDD_software}, plus particulièrement dans la documentation additionnelle de cet article~\cite{salazar_gdd}, nous avons identifié un certain nombre d'\'el\'ements de \emph{Mechanics} typiquement présents dans un GDD.
Nous avons constaté que les éléments étaient répartis dans des catégories possédant elles-mêmes certaines caractéristiques communes. 
Nous avons ainsi extrait les catégories suivantes :

\begin{itemize}
    \item \emph{Player Character}
    \item \emph{Non Player Character}
    \item \emph{Enemy}
    \item \emph{Final Enemy}
    \item \emph{Help Object}
    \item \emph{Extra Object}
    \item \emph{Other Object}
\end{itemize}


\gt{J'ai oté <<très larges>> car <<très larges et spécifiques>> me semblait contradictoire!}

Ces catégories sont cependant plutôt spécifiques au jeu décrit dans cet exemple de \emph{Game Design Document}, à savoir \emph{Donkey Kong}~\cite{salazar_gdd}.
Donc, ces catégories ne permettent pas forcément de représenter les éléments de \emph{Mechanics} de jeux différents de \emph{Donkey Kong}.
Par la suite, nous avons essayé de définir des catégories et des éléments en fonction de différents \emph{Game Design Documents} et de notre expérience personnelle de l'utilisation des jeux vidéos.
Nous avons alors défini les stéréotypes présents dans les figures~\ref{fig.GG1} et~\ref{fig.GG2},
figures qui renvoient vers une modélisation plus en détail présentée dans l'Annexe~\ref{AnnexeA}.

\gt{Ci-haut: il faut expliquer pourquoi ces catégories que tu
identifies ne se retrouvent pas dans les figures 5.1 et 5.2 --- donc
expliquer (?) qu'elles correspondent à des détails de l'annexe (?).}

\gt{Ci-haut: Si ce sont des noms de classes, pr\'esents dans le
mod\`ele/profil UML, alors il faut utiliser la convention des noms de
classe UML, i.e., un seul <<mot>> en CamelCase: PlayerCharacter,
FinalEnemy, etc.}
\eh{J'ai enlevé le txt teletype dans cet itemize car ce sont des éléments présents dans l'exemple de GDD Donkey Kong de l'article (sous forme textuel) et non pas des éléments du profil qui sont cités}

\gt{Par contre, comme ce sont des termes en anglais, il vaut mieux les
mettre en italiques.}

Les classes UML fonctionnant avec un système d'héritage,
un \'el\'ement de \emph{Mechanics} hérite donc des attributs énoncés dans les catégories parentes.
\begin{comment}
Phrase inutile!
C'est ainsi que nous avons décidé d'utiliser un diagramme de classes afin de représenter les \'el\'ements de \emph{Mechanics} comme des classes et de spécifier leurs caractéristiques dans les attributs.
\end{comment}
%
\gt{Je crois que les phrases qui suivent sont redondantes, n'apportent
rien de plus. Sinon, il faut les reformuler, car je ne comprends pas
bien ce qu'elles introduisent de nouveau.}
\eh{Supprimées}

\begin{figure}
    \centering
    \includegraphics[width=5cm]{10_img/chap5/sniper2.PNG} 
    \caption{Une classe \texttt{Sniper1} utilisant le stéréotype
    \texttt{SR}, et donc qui h\'erite des attributs (directs et indirects) associ\'es à son stéréotype \texttt{SniperRifle}.}
    \label{fig.sniper}
\end{figure}

\gt{Idem pour ce qui suit!}
%

\gt{Dans la figure 5.4, boite pour Sniper1. a. Changer <<stereotype>>
par <<SniperRifle>>. b. Il devrait y avoir une ligne (sous-boite de
couleur différente?) qui sépare le nom de la classe de liste des
attributs.}

La figure~\ref{fig.sniper} pr\'esente un exemple d'une classe \texttt{Sniper1}, un élément de \emph{Mechanics} d'un GDD pour un jeu qui serait d\'efini avec le profil \emph{Game Genesis}.
La classe \texttt{Sniper1} est stéréotypée avec \texttt{SniperRifle}, donc elle
hérite des attributs des divers stéréotypes de la hiérarchie d'héritage de classes.
Une instance de la classe \texttt{Sniper1} possèdera donc les attributs des stéréotypes \texttt{Item}, \texttt{Wearable}, \texttt{Weapon} et \texttt{SR}.

\gt{Ci-haut et dans la figure: non, ce n'est pas ainsi qu'il faut
pr\'esenter cela il me semble. Plusieurs remarques. 1) Ce serait bien
de toujours distinguer, par les couleurs, la sp\'ecification du profil
(bleu!) de son utilisation (vert!). Or, la sp\'ecification de Sniper1
est une utilisation.  2) Si tu mets directement les d\'etails dans la
boite Sniper1, alors c'est comme si l'usager les sp\'ecifiait
lui-m\^eme. Or, ici, de ce que je comprends, tu essaies d'expliquer
qu'il y a h\'eritage implicite.  Une possibilit\'e serait de mettre en
vert une boite avec <<SR>> et Sniper, sans attributs, puis juste \`a
cot\'e de mettre une boite (autre couleur?) avec juste Sniper comme
nom mais o\`u tous les attributs sont maintenant pr\'esents ---
justement \`a cause de l'h\'eritage {\bf implicite} associ\'e \`a ta
hi\'erarchie de st\'er\'eotypes.}

\gt{En d'autres mots, il me semble que tu dois insister plus sur le
fait que la sp\'ecification d'un st\'er\'eotype introduit un nouveau
{\bf concept} --- une nouvelle classe ---, lequel concept peut
\'evidemment poss\'eder des attributs --- indiqu\'es directement (dans
sa boite) ou indirectement (via h\'eritage).  Une fois ce concept
d\'efini, on peut ensuite l'utiliser pour d\'efinir de nouvelles
classes, dont les objets poss\`edent tous les attributs associ\'es au
concept --- peu importe d'o\`u proviennent ces attributs (directs ou
indirects).}

\gt{De plus, puisque tu as un exemple d'utilisation du st\'er\'eotype
SR, tu devrais aussi donner un exemple de sp\'ecification/utilisation
des valeurs \'etiquet\'ees pour sp\'ecifier des attributs.}



\subsection{Les interactions}

\begin{figure}
    \centering
    \includegraphics[width=5cm]{10_img/chap5/metaclass_association.PNG} 
    \caption{La spécification du stéréotype \texttt{Interaction} de \emph{Game Genesis}, obtenu par \emph{extension} de la m\'etaclasse \texttt{Association}.}
    \label{fig.meta_assoc}
\end{figure}

Les éléments de \emph{Mechanics} ne sont pas tous des éléments isolés et une expérience de jeu ne serait rien sans les interactions que le joueur peut avoir avec les divers \'el\'ements du jeu.

La Figure~\ref{fig.meta_assoc} représente la spécification des stéréotypes présents dans la catégorie \texttt{Interaction}.
Contrairement aux autres stéréotypes du profil \emph{Game Genesis} qui étendent les classes, celui d'\texttt{Interaction} étend plutôt la métaclasse \texttt{Association}.

\begin{table}
\begin{center}
\begin{tabular}{|l|l|}\hline
\texttt{Use} &
\texttt{Drop}
\\\hline
\texttt{Move}&
\texttt{Destroy}
\\\hline
\texttt{Trade} &
\texttt{Stack}
\\\hline
\texttt{Attach}&
\texttt{Attack}
\\\hline
\texttt{Wear}&
\\\hline
\end{tabular}
\end{center}
\caption{Les interactions de Salazar \emph{et al.}~\cite{salazar_gdd} intégrées dans \emph{Game Genesis}.}
\label{table.interactions}
\end{table}


Des interactions fréquemment rencontrées sont présentées dans le gabarit de GDD de Salazar \emph{et al.}~\cite{salazar_gdd}, qui les présente sous forme de règles d'interaction.
Ces règles sont composées de divers éléments, dont les suivants :
\begin{itemize}
    \item Deux \'el\'ements (ou plus) de \emph{Mechanics};
    \item Une interaction entre ces \'el\'ements;
    \item Une contrainte appliquée à cette interaction (optionnelle).
\end{itemize}
Les interactions de Salazar \emph{et al.} que nous avons intégrées à \emph{Game Genesis} sont présentées dans le tableau~\ref{table.interactions} --- voir aussi la Figure~\ref{A-Interaction}.


\gt{Il faut aussi que tu montres clairement, parce que c'est dans le
contexte d'un profil UML, qu'il s'agit d'extension de la Metaclass
fAssociation d'UML --- comme \'evoqu\'e l'autre fois (quand on s'est
rencontr\'es? je ne suis plus certain.}

\gt{Idem pour les st\'er\'eotypes de classes de la section
pr\'ec\'edente: on devrait voir ici -- et possiblement aussi dans
l'annexe --- qu'il s'agit d'extension de la Metaclass Class!}

\subsection{Quelques contraintes}

Dans un profil UML, il est fréquent d'imposer des contraintes d'utilisation sur les \'el\'ements du profil afin de le rendre plus précis et d'éviter des problèmes de compréhension ou d'incohérence du modèle.
Dans \emph{Game Genesis}, il est difficile d'établir un grand nombre de contraintes, le profil devant s'appliquer à de nombreux types de \emph{gameplay}.
\begin{comment}
Je ne comprends pas trop cette phrase
C'est en essayant de laisser assez de liberté aux \emph{game designers} que certaines contraintes ne peuvent pas être appliquées ou sont situationnelles.
\end{comment}

Voici quand m\^eme une br\`eve liste de contraintes envisageables dans un profil tel que \emph{Game Genesis} :
\begin{itemize}
    \item Une interaction entre \texttt{Player} et \texttt{Weapon} ne peut être que \texttt{Grab} ou \texttt{Equip};
    \item Une interaction entre \texttt{Player} et \texttt{Equipment} ne peut être que \texttt{Grab} ou \texttt{Equip};
    \item Une interaction entre \texttt{AddOn} et \texttt{Weapon} ou \texttt{Equipment} ne peut être que \texttt{Attach};
\end{itemize}

\gt{Je ne comprends pas la contrainte sur AddOn~:(} 

Certaines contraintes peuvent être situationnelles comme celles ci-dessous :
\begin{itemize}
    \item Si le jeu est \texttt{PvP} (\emph{Player versus Player}) alors \texttt{Player} \texttt{<<attack>>} \texttt{Player} est possible,
    \item Si le jeu n'est pas \texttt{PvP} alors \texttt{Player} \texttt{<<attack>>} \texttt{Player} n'est pas possible.
\end{itemize}


\gt{Ci-haut: dans ce cas, si tu comptes \'enoncer une telle
contrainte, il faudrait que ton profil inclut un attribut qui permet
d'indiquer si le jeu et PvP ou non.}
\eh{Un tel attribut serait trop contraignant dans le cadre d'une conception de jeux. Certains jeux sont PVP, d'autres PVE, d'autres ont des zones uniquement PVP et/ou PVE, certains crééent plusieurs serveurs qui sont soit PVP soit PVE, etc.
Le PVP ou PVE sont des mécaniques de gameplay }

\section{Conclusion}
%
%\GT{Dire que tu as <<réussi>> est trop fort!}
%
Dans ce chapitre, nous avons tenté d'exprimer les concepts de base d'un profil UML pour la rédaction d'un \emph{Game Design Document}.
Dans son état actuel, ce profil, \emph{Game Genesis}, est cependant plus une <<preuve de concept>> qu'une version complète qui pourrait couvrir tous les besoin d'un \emph{game designer}.
\emph{Game Genesis} est incomplet car :
\begin{itemize}
    \item seuls les aspects \emph{Mechanics} sont traités;
    \item les catégories sont non exhaustives;
    \item le profil est spécifié de façon non formelle, aucun outil UML <<officiel>> n'ayant été utilisé pour le spécifier.
\end{itemize}

Il serait quand même intéressant de voir comment \emph{Game Genesis} peut être utilisé dans le cadre de la rédaction d'un \emph{Game Design Document} pour un jeu existant.
C'est ce que nous faisons dans le prochain chapitre.




%%%%%%%%%%%%%%%%%%%%%%%%%%%%%%%%%%%%%%%%%%%%%%%%%%%%%%%%%%%%%%%%%%%%%%%%%%
\begin{comment}
\chapter{Un langage de modélisation pour l'établissement d'un Game Design Document}

\section{Le concept}
\subsection{Définition}
\subsubsection{Quoi ?}
Un langage permettant de modéliser et stocker des idées lors des phases de Breakthrough et de Conception d'un projet de jeu vidéo. La modélisation peut être graphique et/ou textuelle avec application des modifications en parallèle. \\
Les informations peuvent contenir tout le nécessaire pour exprimer les idées (textes, informations numériques, chemins de fichiers...). Les champs peuvent être personalisables pour permettre de la souplesse aux utilisateurs.

\subsubsection{Pour quoi faire ?}
\paragraph{Des outils de modélisation existent pour tous les domaines reliés au développement de logiciels. Ils sont souvent spécifiques à un corps de métier afin de pouvoir proposer un maximum de fonctionnalités spécifiques sans devenir trop compliqué et en utilisant un vocabulaire précis qui correspond au corps de métier concerné.}

\paragraph{Il y a peu ou pas de langages de modélisation plus généraux pour des domaines multi-métiers. Le but est de pouvoir modéliser la réflexion créative en fournissant un élément visuel permettant de mind-mapper les idées, les stocker et les réutiliser. \\
Il faut que la modélisation soit assez souple pour pouvoir répondre aux besoins de chacun des corps de métier d'où le fait que les éléments et attributs peuvent avoir des identifiants spécifiques définis librement par l'utilisateur.}

\subsubsection{Pour quelles raisons ?}
\paragraph{Les supports de réflexion actuellement utilisés : cahier des charges, réunions, notes écrites, mails, minds-maps... L'organisation de ces différents supports dans un ensemble cohérent est tr;s compliqué. Dans un cahier des charges il est compliqué de classer les idées à la volée. Un mind-map nécessite une numérisation ou une retranscription sur un outil de mind-mapping qui sont toutes les deux des techniques non péreines et risquées dans la conservation des données. Des notes écrites peuvent se perdre et n'ont pas de durabilité sur le long terme. Des mails sont péreins mais il est difficile de les organiser pour le stockage de l'information.}
\paragraph{Un langage de modélisation graphique et textuel permettrait de mind-mapper les idées à la volée sous forme de cubes contenant les données nécessaires. La hiérarchisation des éléments permettrait de gérer des héritages et des relations ainsi que d'éviter la répétition trop abondante des mêmes informations. Les faces des cubes permettrait d'isoler les informations nécessaires à chacun des corps de métier.}
\end{comment}

\chapter{Un exemple d'application du profil \emph{Game Genesis}~: Le jeu PUBG}

PUBG est un jeu vidéo, sorti en 2017, qui compte d\'ej\`a des millions d'adeptes à travers le globe. Développé par \emph{PUBG Corporation} (filiale de B\emph{luehole, Inc.}, plus récemment de \emph{Krafton Game Union}), PUBG est un des premiers jeux \emph{standalone}%
\footnote{Standalone : Logiciel informatique qui peut s'exécuter de manière indépendante}
%
, avec  \emph{Fortnite} (\emph{Epic Games}),
permettant aux joueurs de participer à un <<\emph{Last man standing game}>>. Ces deux jeux ont été les premiers \emph{standalone} à populariser le \emph{Battle Royale} et à le mettre à la portée de tous sans devoir développer ou installer du contenu additionnel dans des jeux existants.

\gt{Des mods!?}


\section{L'origine des jeux \emph{Battle Royale}}

\gt{Ci-bas: Donner aussi une r\'ef\'erence bibliographiqu pour le roman.}

\gt{Ci-bas: je ne comprends pas <<50 classes de troisi\`eme>>!?}

Les jeux de type \emph{Battle Royale} trouvent leurs racines dans le roman <<\emph{Battle Royale}>>~\cite{takami2003battle} et son adaptation cinématographique.%
%
\footnote{\url{https://www.imdb.com/title/tt0266308}}
%
En gros, l'histoire va comme suit.
Un programme militaire de simulation de combat a lieu dans une République socialiste d'Extrême-Orient complètement coupée du monde extérieur, extrêmement stable politiquement, o\`u les habitants n'ont aucun droit civique. Le programme militaire d\'ebute par un tirage au sort annuel de 50 classes d'élèves de troisième année. La classe tirée au cort est alors déplacée chacune vers une zone de combat. Au début de l'expérience, tous les élèves sont réunis afin de recevoir un briefing rapide, puis se voient attribuer un sac contenant un seul objet (par ex., arme à feu, arme blanche, fourchette, corde de luth, etc). Les \'el\`eves sont ensuite transport\'es et livrés à eux-mêmes dans une zone donnée en ayant pour seule consigne qu'aucune règle n'est imposée\ldots\ et qu'un seul survivant par groupe pourra être gagnant de l'expérience;  les autres devront être exterminés, et ce par n'importe quel moyen. Le champion obtient alors le droit de vivre aux frais de l'État pour le reste de ses jours et la reconnaissance comme <<~Héros du pays~>>.

\section{L'essor de la popularité du \emph{Battle Royale}}
Bien que connu depuis l'ann\'ee 2000, l'intérêt pour le principe du \emph{Battle Royale} n'explose que plus tard, avec le succès de la saga \emph{Hunger Games}%
%
\footnote{\url{https://www.imdb.com/title/tt1392170/}}
%
, mettant en place l'histoire de Katniss Everdeen, une jeune adulte qui participe de son plein gré aux \emph{Hunger Games}. Dans un état totalitaire séparé en castes regroupées dans des districts, deux enfants ou adolescents sont choisis au hasard dans chaque district afin de devenir les Tribus. Ils sont alors réunis dans la capitale et lâchés dans une arène afin de participer à une télé-réalité de match à mort diffusée partout dans le pays.

Le succès des \emph{Hunger Games} a amené les développeurs du jeu \emph{Arma II} à créer un mode de jeu basé sur les mêmes règles. Ce mode sera alors appelé \emph{DayZ} et deviendra par la suite un jeu \emph{standalone}. Un mode voit également le jour : \emph{Battle Royale}. Développé par Brendan Greene tout d'abord pour \emph{Arma II} puis pour \emph{Arma III}, ce mode gagne suffisament en popularité pour qu'il donne naissance au projet de jeu \emph{PlayerUnknown's Battlegrounds} (PUBG) --- \emph{PlayerUnknown} étant le pseudonyme en ligne utilisé par Brendan Greene.

\section{Les règles de PUBG}
Les r\`egles du jeu PUBG vont comme suit.
Un avion parcourt une ligne droite au-dessus d'un territoire, et les 100 joueurs de la partie doivent sauter de cet avion afin d'être parachutés à l'endroit de leur choix --- à condition qu'il soit à portée de parachutage. Une fois arrivés au sol, les joueurs partent à la recherche d'armes et d'équipements dans des bâtiments et doivent s'exterminer jusqu'à ce qu'un seul joueur soit vivant et gagne ainsi la partie. Il est également possible de jouer en \'equipes de deux ou quatre joueurs, o\`u la dernière équipe encore vivante est l'\'equipe gagnante. Afin de limiter la durée d'une partie, une zone circulaire est définie sur la carte une fois que tous les joueurs ont atterri, et cette zone se réduit au cours de la partie. Les joueurs en dehors de la zone subissent une certaine quantité de dégâts s'ils restent \`a l'ext\'erieur de la zone circulaire, et ces dégâts augmentent au fur et à mesure que le cercle se resserre.

\section{Le \emph{Battle Royale} dans le monde du jeu vidéo}
Ces dernières années, une grande quantité de jeux vid\'eos de type \emph{Battle Royale} ont vu le jour. De \emph{Fortnite} (\emph{Epic Games}) à PUBG, en passant par \emph{Apex Legends} (\emph{Respawn Entertainement}), \emph{Ring of Elysium} (\emph{Tencent Games}) pour ne citer que les plus importants. Des dizaines de jeux voient le jour ou se réinventent afin de coller à la mode du \emph{Battle Royale}. Les plus grandes licences de FPS (\emph{First-Person Shooter}) s'alignent également à cette tendance, et c'est ainsi que voient le jours les modes de jeu \emph{Z1BR} (\emph{H1Z1}), \emph{Firestorm} (\emph{Battlefield V}), \emph{Blackout} (\emph{Call of Duty Black Ops IV}). Mais cette offre répond à une demande phénoménale du public pour ce mode de jeu qui s'impose dans le marché à la même place que les jeux de type MOBA (\emph{Multiplayer Online Battle Arena}), qui étaient des grands favoris du public depuis une dizaine d'années, notamment avec les jeux \emph{League Of Legends} ou \emph{Dota 2}. Avec le temps, les jeux de \emph{Battle Royale} tentent de se réinventer et de trouver de nouveaux publics en gardant le principe du \emph{Battle Royale}, mais en explorant d'autres univers. C'est ainsi que naissent \emph{Last Tide} et son univers aquatique immergé, \emph{Fall Guys : Ultimate Knockout} et son univers coloré où de petits personnages à la physique étrange se battent pour une couronne, \emph{Cuisine Royale}, qui était à l'origine une blague de développeurs mais qui a rencontré un immense succès, dans lequel les personnages se battent à l'aide d'ustensiles de cuisine.

\section{Une description (partielle) des \'el\'ements de \emph{Mechanics} du jeu PUBG}
\subsection{Les informations d\'ecrivant le jeu PUBG}
Les informations concernant les objets de PUBG sont difficile à trouver.
Cependant, certains site internet se sont spécialisés dans l'extraction (\emph{mining}) d'informations à partir des fichiers du jeu.
C'est le cas du \emph{Gamepedia} spécialisé pour PUBG (\url{https://pubg.gamepedia.com/}),
un Wiki régulièrement mis à jour et possédant assez d'informations pour permettre d'identifier les objets du jeu et leurs caractéristiques.
Les informations disponibles sont toutefois souvent sujettes à modifications au fur et à mesure des mises à jour du jeu.
Cependant, l'exactitude des informations n'a pas une importance cruciale dans le cadre du pr\'esent travail.


\subsection{\emph{Game Genesis} appliqué à la modélisation de PUBG}

\begin{figure}
    \centering
    \includegraphics[width=14cm]{10_img/chap6/profile_base.PNG}
    \caption{Les stéréotypes de \emph{Game Genesis} utilisés dans l'exemple}
    \label{fig.racine_stereo}
\end{figure}

\begin{figure}
    \centering
    \includegraphics[width=14cm]{10_img/chap6/profile_evo.PNG}
    \caption{Les stéréotypes modifiés pour l'utilisation pour PUBG.}
    \label{fig.racine_stereo}
\end{figure}

Les stéréotypes présents dans \emph{Game Genesis} permettent de classifier les \'el\'ements de \emph{Mechanics} pour la rédaction d'un GDD.
Afin de mieux comprendre le fonctionnement de \emph{Game Genesis}, nous avons extrait une partie du profil et la pr\'esentons dans la Figure~\ref{fig.racine_stereo}.
Cette partie du profil inclut les stéréotypes qui sont utilisés dans les exemples présentés dans la suite du chapitre.
De nombreux attributs ont été ajoutés dans les stéréotypes.
Ceux-ci correspondent aux attributs qui auraient pu être ajoutés par un \emph{game designer} dans le profil.
Ajouter de tels attributs permet d'inclure automatiquement les attributs dans les classes créées à partir du profil UML par l'héritage.

\subsection{La modélisation d'une \AVERIFIER{classe de joueurs avec leurs armes}}


\begin{figure}[H]
    \centering
    \includegraphics[width=14cm]{10_img/chap6/object_joueur.PNG}
    \caption{La modélisation d'une instance de joueur avec son arme.}
    \label{fig.player+weapon+equip}
\end{figure}


\gt{Dans la figure: tu ne peux pas avoir le nom de classe
\texttt{Player} qui soit en plus st\'er\'eotyp\'e par
<<\texttt{Player}>>.}
\eh{Remplacé par Soldier}

\gt{De plus: Pourquoi Player68?  Comme indiqu\'e ailleurs, avec les
st\'er\'eotypes, tu d\'efinis des classes, et non des instances. Or,
quand on voit <<Name: Player68>>, cela donne vraiment l'impression que
c'est une instance qui est d\'efinie, et non une classe.}

\gt{Et aussi: si tu veux sp\'ecifiers classes st\'er\'eotyp\'es via
des valeurs \'etiquet\'ees, alors ce n'est pas la bonne syntaxe.  Il
faut plut\^ot utiliser la syntaxe 'attribut = valeur' --- le tout (il
me semble) entre accolades.  Revoir les articles UML \`a ce sujet.}

\gt{Ci-bas: il faut rendre plus clair, plus explicite, le fait que
c'est une classe de joueurs qui est mod\'elis\'ee, et non un joeur
sp\'ecifique. Parce qu'un st\'er\'eotype sert \`a d\'efinir/d\'ecrire
des {\bf classes} et non des {\bf instances}. Les st\'er\'eotypes sont
utilis\'es dans un mod\`ele de jeu, et non dans une instance
sp\'ecifique du jeu en cours d'ex\'ecution.}

La Figure~\ref{fig.player+weapon+equip} présente la modélisation d'un joueur.
Celui-ci est équipé d'un casque \texttt{Tier3} et d'une veste \texttt{T3}.
Il est armé d'un fusil \emph{sniper} de type \texttt{Kar98k} lui même équipe d'une lunette \texttt{Scope6x} et d'une ceinture de munitions.

\gt{Si possible, il serait pr\'ef\'erable que Equip soit attach\'e \`a
un endroit diff\'erent de Wear.  C'est ok il me semble de mettre
plusieurs fois des instances de la m\^eme association sur une m\^eme
ligne, mais moins de mettre une association compl\`etement
diff\'erente!}

\gt{De plus, il me semble que dans les exemples vus \`a diff\'erents
endroits, le nom d'une association st\'er\'eotyp\'ee doit aussi \^etre
entre chevrons, <<Wear>>, <<Equip>>, etc.}

Comme décrit \`a la Section~\ref{sect.gg_what}, nous avons utilisé des classes afin de représenter les éléments du jeu, alors que des associations entre classes représentent les interactions possibles entre éléments du jeu.
Sur ces classes et associations, nous avons appliqué les stéréotypes du profil \emph{Game Genesis} afin de pouvoir classifier et spécialiser les différents éléments modélisés.



\subsection{La modélisation d'une instance d'attaque d'un joueur sur un autre}
\begin{figure}[H]
    \centering
    \includegraphics[width=8cm]{10_img/chap6/playervsplayer.PNG}
    \caption{Un exemple d'interaction \texttt{Player} <<\texttt{attack}>> \texttt{Player}.}
    \label{fig.attack}
\end{figure}

\GT{Figure: je vais revoir cette figure ult\'erieurement! Mais tu
r\'ef\`eres \`a \texttt{Player68} (ou 25) alors que, formellement, n'y
a pas de classe qui porte ce nom.}

La Figure~\ref{fig.attack} présente une sp\'ecification d'une interaction possible entre joueurs, \`a savoir un joueur qui en attaque un autre.
L'élément \texttt{Player68} est celui modélisé \`a la section précédente.

Le \emph{game designer} a modifié le stéréotype \texttt{attack} afin que celui-ci comprenne le calcul des dégâts d'un joueur sur un autre.
Afin de calculer ces dégâts, nous avons besoin d'un certain nombre d'informations.
Dans un FPS, il est possible d'avoir plusieurs types de dégâts en fonction de l'emplacement de l'attaque.
C'est le cas dans PUBG qui différencie les dégâts en trois types : tête, torse, autres.
Ensuite le calcul des dommages s'effectue en prenant en compte les protections portées par le joueur adverse.
Afin d'exprimer le calcul des dégâts nous avons mis en place la formule suivante :

\begin{equation*}
\begin{split}
Damages& = Attacker(Weapon(Damages(Type(Value)))\\
Protection& = Defender(Protection(DamageReduction))\\
Calcul& = Defender(health) - (Damages - Protection)
\end{split}
%\label{calc.damages}
\end{equation*}

\GT{Ci-haut pour l'\'equation: si par la suite tu ne r\'ef\`eres pas
explicitement \`a cette \'equation, alors pas besoin de lui donner un
num\'ero et un label.}


\GT{Ci-haut et ci-bas: pas besoin de mettre en guillemets si tu
utilises la police appropri\'ee~: \texttt{Attack}.}

Nous retrouvons cette formule de calcul dans la Figure~\ref{fig.attack}.
Elle est présente dans l'association entre les deux joueurs sur laquelle un stéréotype \texttt{Attack} est appliqué.
On considère alors que le \emph{game designer} a modifié le stéréotype de \emph{Game Genesis} afin que toutes les actions d'attaque soient régies par cette formule.

Dans cette figure, il est spécifié que le joueur \texttt{Player68} effectue un tir à la tête sur le joueur \texttt{Player25}.
Le calcul des dommages renvoie un résultat de 118.5 points de dégats.
Le joueur \texttt{Player25} ne possède cependant que 100 points de vie (\texttt{Health}).
On peut donc considérer qu'à la suite de cette action d'attaque, le joueur \texttt{Player25} est éliminé,
ce qui peut être exprimé par les contraintes suivantes~:

{
\footnotesize
\begin{framed}
    Player(health) : is max 100.\\
    (IF Player(health) < 0)\\
    \{\\
    IF Gamemode is "solo" ==> Player is dead and eliminated.\\
    IF Gamemode is "duo" OR "squad" ==> Player is knocked.\\
    \}
\end{framed}
}

\section{Conclusion}


\begin{conclusion}

%le but => Faciliter la modélisation des mecha dans le proc de pre conception d'un jv
Un projet de développement informatique est une tâche de longue haleine rythmée par de nombreux changements en cours de développement.
La conception et la description des concepts est essentielle au bon déroulement du projet ainsi qu'à sa réussite.

C'est encore plus marqué dans le domaine du développement de jeux vidéos.
Le développement de jeux vidéos implique non seulement des développeurs mais également beaucoup d'autres corps de métier comme des artistes faisant des planches pour représenter les éléments, des musiciens créant la bande originale du jeu, des modéleurs 3D et animateurs qui s'occupent de créer les éléments présents dans le jeu, des scénaristes mettant en place le cadre et l'histoire du jeu, etc.
Réussir à apporter toutes les informations nécessaires à tous ces corps de métier est un défi que le~\emph{game designer} affronte au quotidien.

Les documents de~\emph{game design} deviennent donc un pilier des projets de développement de jeux vidéos.
Leur rédaction est compliquée, chronovore et nécessite une rigueur exemplaire des~\emph{game designer}.
Cependant aucun template préconçu n'existe pour répondre à tous les besoins des~\emph{game designer}.
Chaque projet est différent, chaque jeu possède des spécificités, chaque équipe de développement est différente et cela empêche l'établissement d'un modèle générique de~\emph{game design document}.

Dans la littérature nous avons pu trouver de nombreuses listes de bonnes pratiques de design, chacune essayant de généraliser leurs conseils pour tous les genres de jeu vidéo.
C'est le cas du~\emph{Framework MDA} qui propose un découpage d'un jeu vidéo afin de structurer les différents éléments dans des catégories et lier tous ces éléments pour qu'ils fonctionnent ensemble.

Définir tous les éléments d'un système et décrire leurs interactions est une approche déjà présente dans le développement logiciel classique.
C'est ce que l'on retrouve à travers la modélisation UML dans un projet informatique classique.
Cependant UML est un langage spécifique au domaine du développement informatique et il est donc compliqué pour les autres corps de métier d'acquérir les connaissances nécessaires à sa compréhension.
Il est cependant possible de créer des profils UML afin d'adapter les modèles UML et les étendre afin d'intégrer un vocabulaire spécifique à un domaine d'activité.

Une fois le vocabulaire adapté à la conception de jeux vidéos UML a la capacité d'apporter beaucoup de ses points forts à la conception.
C'est un langage formel, permettant la description de systèmes et de processus, efficace, extensible et largement documenté.
Les outils qui l'accompagnent sont fiables et respectent une rigueur qui permet la réutilisation des modèles de façon linéaire et en respectant le contenu des données.
Sa formalisation permet également d'assurer la pérennité des modèles dans le temps et leur cohérence peu importe le nombre de modifications et la durée de leur utilisation.

C'est ce que nous recherchions dans notre problématique : apporter une description fiable, modifiable, compréhensible et précise dans la pré-conception d'un jeu vidéo et la rédaction de son~\emph{Game design document}.

C'est dans cette optique que nous avons mis en place~\emph{Game Genesis}, un profil UML permettant d'intégrer le vocabulaire adapté à la conception de jeux vidéos dans des modèles.

Nous avons étudié différents~\emph{Game design document} afin de définir des caractéristiques communes à ces documents.
Nous nous sommes concentrés sur la modélisation des éléments de~\emph{Mechanics} présents dans ces documents afin d'établir une liste d'éléments qui se retrouvent dans de nombreux genres et donc de nombreux projets de développement de jeux vidéos.
Une fois cette liste établie nous avons établis une liste de stéréotypes correspondant à ces éléments.
Il a fallut établir une structure hiérarchique de ces éléments afin de les classifier dans des catégories de regroupement conceptuel.
Afin de représenter cette hiérarchie nous avons utilisé le mécanisme d'héritage d'UML.
Cela nous a permit de définir des caractéristiques dans les catégories et les éléments qui leur sont associés héritent des attributs.

Afin de décrire les interactions entre les éléments de~\emph{Mechanics} nous avons utilisé les Associations d'UML.
Les stéréotypes de~\emph{Game Genesis} comprennent une catégorie~\texttt{Interaction} qui s'appliquent sur les associations UML.
Ces stéréotypes décrivent les interactions entre les éléments et peuvent porter les attributs et méthodes nécessaires.

Nous avons ensuite testé ce profil dans un exemple d'application réel à travers le jeu PlayerUnknown's BattleGround.
Nous avons réussi à modéliser les concepts généraux d'une partie de ce jeu en utilisant un diagramme de classes sur lequel était appliqué~\emph{Game Genesis}.
Par ce modèle nous avons réussis à décrire une partie des éléments de~\emph{Mechanics} du jeu et leurs interactions.

Cependant~\emph{Game Genesis} ne se concentre que sur la description des éléments de~\emph{Mechanics}.
Ce n'est qu'une partie du design de jeux vidéos, de la rédaction de~\emph{Game design document} et du~\emph{Frameowrk MDA}.
Nous avons décidé de nous concentrer sur ces éléments uniquement car le travail de représentation de tous les éléments du framework était trop complexe à réaliser.

La représentation des éléments dans un diagramme de classes ouvre de nombreuses possibilités pour la rédaction de~\emph{Game design document} et pour le développement de jeux vidéos.
En effet il est possible de réutiliser les modèles facilement, UML étant un langage informatique il est possible de le manipuler et de générer d'autres documents.
En réutilisant les modèles il serait possible d'extraire des informations des modèles afin de générer le début d'un~\emph{Game Design Document} décrivant les éléments de~\emph{Mechanics} du jeu en question.
Il serait également possible de générer un squelette d'application à partir des modèles, un diagramme de classe étant tout indiqué pour cette transformation.
Nous pourrions imaginer un outil de versionning permettant de faire un différentiel entre les versions des modèles.
Il serait alors possible de stocker les justifications d'une modification effectuée et documenter ces modifications pour le futur.

Finalement nous espérons que cette recherche permettra d'apporter un outil adapté au domaine du développement de jeux vidéos. La finalité étant d'apporte une représentation graphique des éléments de~\emph{Mechanics} permettant de structurer et de documenter le processus de pré-conception et d'accompagner les~\emph{game designer} tout au long d'un projet de conception.


%la nature et l’envergure du travail
%les sujets traités => Dev jv, gdd, mda, profils
%les problèmes à résoudre => formaliser une représentation des Mecha, faciliter la representation, 
%les objectifs fixés => apporter un outil fiable, langage adapté aux JV, malléable pour tous les gameplay, 
%les méthodes utilisées => profil UML,Établir un profil UML pour représenter les éléments de Mechanics dans le processus de pré-conception d'un JV
%la démarche adoptée => GDD (exemples et bonnes pratiques), MDA (section Mecha), Profil UML
%les résultats les plus saillants => Exemple d'application ok, possibilité de decrire les mecha et leurs relations efficacement
%les limites et les conclusions => Représentation des Mecha (no dyna no aesth)
%les recommandations et les pistes de recherche => utilisation des modeles générés pour : générer un squelette de GDD, générer un squelette de code, stocker les données pour le versionning


\end{conclusion}



% Utilisez l'environnement  conclusion pour rédiger votre conclusion


\chapter*{ANNEXE A}
\addcontentsline{toc}{chapter}{ANNEXE A}  

\section*{Item}
\addcontentsline{toc}{section}{Item}  
\begin{figure}[H]
    \centering
    \includegraphics[width=14cm]{10_img/chap5/01_00_item.PNG} 
    \caption{TITRE}
\end{figure}
\begin{figure}[H]
    \centering
    \includegraphics[width=14cm]{10_img/chap5/01_01_wearable.PNG} 
    \caption{TITRE}
\end{figure}
\begin{figure}[H]
    \centering
    \includegraphics[width=14cm]{10_img/chap5/01_01_01_weapon.PNG} 
    \caption{TITRE}
\end{figure}
\begin{figure}[H]
    \centering
    \includegraphics[width=14cm]{10_img/chap5/01_01_02_equipment.PNG} 
    \caption{TITRE}
\end{figure}
\begin{figure}[H]
    \centering
    \includegraphics[width=14cm]{10_img/chap5/01_01_03_jewerly.PNG} 
    \caption{TITRE}
\end{figure}
\begin{figure}[H]
    \centering
    \includegraphics[width=14cm]{10_img/chap5/01_01_04_tool.PNG} 
    \caption{TITRE}
\end{figure}
\begin{figure}[H]
    \centering
    \includegraphics[width=14cm]{10_img/chap5/01_03_addon.PNG} 
    \caption{TITRE}
\end{figure}
\begin{figure}[H]
    \centering
    \includegraphics[width=14cm]{10_img/chap5/01_04_usable.PNG} 
    \caption{TITRE}
\end{figure}
\begin{figure}[H]
    \centering
    \includegraphics[width=14cm]{10_img/chap5/01_05_craft.PNG} 
    \caption{TITRE}
\end{figure}
\begin{figure}[H]
    \centering
    \includegraphics[width=14cm]{10_img/chap5/01_06_currency.PNG} 
    \caption{TITRE}
\end{figure}


\section*{Animate}
\addcontentsline{toc}{section}{Animate}  
\begin{figure}[H]
    \centering
    \includegraphics[width=14cm]{10_img/chap5/02_00_animate.PNG} 
    \caption{TITRE}
\end{figure}
\begin{figure}[H]
    \centering
    \includegraphics[width=14cm]{10_img/chap5/02_00_01_character.PNG} 
    \caption{TITRE}
\end{figure}
\begin{figure}[H]
    \centering
    \includegraphics[width=14cm]{10_img/chap5/02_00_02_creature.PNG} 
    \caption{TITRE}
\end{figure}



\section*{Character Sheet}
\addcontentsline{toc}{section}{Character Sheet}  
\begin{figure}[H]
    \centering
    \includegraphics[width=14cm]{10_img/chap5/03_00_charactersheet.PNG} 
    \caption{TITRE}
\end{figure}
\begin{figure}[H]
    \centering
    \includegraphics[width=14cm]{10_img/chap5/03_00_01_statistic.PNG} 
    \caption{TITRE}
\end{figure}
\begin{figure}[H]
    \centering
    \includegraphics[width=14cm]{10_img/chap5/03_00_02_attribute.PNG} 
    \caption{TITRE}
\end{figure}
\begin{figure}[H]
    \centering
    \includegraphics[width=14cm]{10_img/chap5/03_00_03_information.PNG} 
    \caption{TITRE}
\end{figure}


\section*{LORE}
\addcontentsline{toc}{section}{LORE}  
\begin{figure}[H]
    \centering
    \includegraphics[width=14cm]{10_img/chap5/04_00_lore.PNG} 
    \caption{TITRE}
\end{figure}


\section*{World}
\addcontentsline{toc}{section}{World}  
\begin{figure}[H]
    \centering
    \includegraphics[width=14cm]{10_img/chap5/05_00_world.PNG} 
    \caption{TITRE}
\end{figure}


\section*{Interaction}
\addcontentsline{toc}{section}{Interaction}  
\begin{figure}[H]
    \centering
    \includegraphics[width=14cm]{10_img/chap5/06_00_interaction.PNG} 
    \caption{TITRE}
\end{figure}

%%%%%%%%%%%%%%%%%%%%
% Page liminaires
%%%%%%%%%%%%%%%%%%%%
\bibliographystyle{apalike-uqam}
\bibliography{03_post/bib}

\end{document}
