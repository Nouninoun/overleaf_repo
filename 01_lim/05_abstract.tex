\begin{abstract}
Le développement de jeux vidéos est un domaine en pleine expansion depuis plusieurs années.
Les jeux deviennent de plus en plus complexes et de plus en plus importants en taille.
Les projets deviennent beaucoup plus compliqués, les budgets plus importants et le temps de développement de plus en plus réduit.
Afin d'accélérer toutes les étapes du développement d'un jeu vidéo de nombreux outils émergent dans ce domaine.
Les moteurs de jeu mettent à disposition des environnement clé en main afin d'aider les équipes de développement ils incluent : 
les environnements de développement intégrés, les outils de gestion de projet, les logiciels de création 3D, les environnements de développement d'intelligences artificielles, la gestion des animations, etc.
Cependant un pan complet de la création de jeux vidéos reste encore peu formalisé et peu outillé : le~\emph{game design}.
Pourtant la réflexion sur le~\emph{game design} et la documentation des concepts d'un jeu vidéo sont des étapes cruciales du développement de jeu vidéo.
Sans une solide documentation et des concepts clairement énoncés un projet de développement peut être rapidement voué à l'échec.
Nous avons recherché les bonnes pratiques de rédaction d'un~\emph{Game Design Document} ainsi que les outils actuellement utilisés pour les étapes de~\emph{game design} et de pré-conception.
Dans ce mémoire nous allons proposer~\emph{Game Genesis}, un profil UML pour faciliter la rédaction d'un~\emph{Game Design Document}.
UML introduit : un langage formel, une structure précise, des outils performants et un support de communication efficace.
Le domaine du développement de jeux vidéos est cependant très vaste. 
La taille des équipes de développement, les genres de jeux vidéos, les types de~\emph{gameplay}, les spécificités de~\emph{game design} entraînent une difficulté à anticiper tous les éléments nécessaires à la rédaction de~\emph{Game Design Document}.
La liste des stéréotypes définis dans~\emph{Game Genesis} est donc non exhaustive et les éléments répertoriés sont assez généraux afin de ne pas représenter d'obstacles à la représentation de certains types de jeux vidéos.
Afin de structurer ce profil nous allons faire usage du~\emph{Framework MDA} qui sépare le~\emph{game design} en trois catégories d'éléments :~\emph{Mechanics},~\emph{Dynamics},~\emph{Aesthetics}.
Avec~\emph{Game Genesis} nous proposons une structure pré-existante afin de représenter les~\emph{Mechanics} d'un jeu vidéo.
Nous avons pu établir une liste d'éléments de~\emph{Mechanics} permettant de représenter les jeux vidéos.
L'utilisation de~\emph{Game Genesis} 


MOTS CLÉS :~\emph{Game Design}, jeux vidéos,~\emph{Game Design Document},~\emph{Framework MDA}, profil UML.

\end{abstract}






