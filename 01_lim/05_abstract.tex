\begin{abstract}

\gt{Dans le cadre du mémoire, tu décris <<ce que tu as fait>>. Il est donc
généralement préférable d'utiliser le passé plutôt que le futur.}

\gt{Oops! Comme indiqu\'e hier, j'ai des probl\`emes avec mon MacOS --
au niveau du termial utilis\'e.  J'utilise donc un nouveau
terminal. Le bon point: il me permet de voir correctement les accents,
et donc spontan\'ement j'en ai mis quelques-uns... jusqu'\`a ce que je
r\'ealise que peut-\^etre que toi, sur Windows, tu ne les vois pas
correctement.  Donc, je continue \`a utiliser des accents \LaTeX, mais
si tu vois des caract\`eres bizarres dans le dernier commit ou dans le
r\'esum\'e, c'est a cause de cela.}

\eh{Je travaille sur overleaf, donc les accents \'e et é sont tous deux pris en charge sans soucis}


\gt{Ci-bas: j'ai simplifi\'e un peu, car le r\'esum\'e me semblait un peu long.}

Le développement de jeux vidéos est un domaine en pleine expansion.
Les jeux deviennent plus complexes,
les projets plus compliqués, les budgets plus importants, les d\'elais de développement plus serr\'es.
Afin d'accélérer les étapes du développement d'un jeu, de nombreux outils émergent~: 
 moteurs de jeu, environnements de développement intégrés, outils de gestion de projet, logiciels de création 3D, aide au développement d'intelligences artificielles, gestion des animations, etc.
Cependant, un pan complet de la création de jeux vidéos reste encore peu formalisé et peu outillé : le~\emph{game design}.
%
Pourtant, la réflexion sur le~\emph{game design} et la documentation des concepts d'un jeu vidéo sont des étapes cruciales.
Sans une solide documentation et des concepts clairement énoncés, un projet de développement peut être rapidement voué à l'échec.

Dans le cadre de ce mémoire,
nous avons cherché à identifier les bonnes pratiques de rédaction d'un~\emph{Game Design Document} ainsi que les outils utilisés pour les étapes de~\emph{game design}.
Comme r\'esultat de ce travail, nous proposons~\emph{Game Genesis}, un profil UML pour faciliter la rédaction d'un~\emph{Game Design Document}.

UML introduit un langage formel, une structure précise, des outils performants et un support de communication efficace.
Le domaine du développement de jeux vidéos est cependant très vaste. 
La taille des équipes de développement, les genres de jeux vidéos, les types de mécaniques de jeu (\emph{gameplay}), les spécificités de~\emph{game design} entraînent une difficulté à anticiper tous les éléments nécessaires à la rédaction d'un \emph{Game Design Document}.
La liste des \'el\'ements introduits dans~\emph{Game Genesis} est donc non exhaustive et ces éléments sont assez généraux afin de ne pas faire obstacle à la représentation de certains types de jeux vidéos.

\gt{Ci-haut: un peu trop tot (contexte insuffisant) pour parler de
<<st\'er\'eotypes>>, donc <<\'el\'ements>> devrait suffire.}

Afin de structurer le profil \emph{Game Genesis}, nous avons fait usage du~\emph{framework MDA}, qui sépare le~\emph{game design} en trois aspect~: \emph{Mechanics}, \emph{Dynamics} et \emph{Aesthetics}.
Plus sp\'ecifiquement, avec~\emph{Game Genesis}, nous proposons un profil pour aider \`a mod\'eliser les \'el\'ements de~\emph{Mechanics} d'un jeu vidéo.
%
Nous illustrons l'utilisation de ce profil en mod\'elisant les
\'el\'ements de \emph{Mechanics} de PUBG (\emph{PlayerUnknown's
Battlegrounds}), un jeu vid\'eo populaire sorti en 2017.


MOTS CLÉS :~\emph{Game design}, jeux vidéos,~\emph{Game Design Document},~\emph{framework MDA}, profil UML.

\end{abstract}






