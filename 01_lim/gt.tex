\GT{La l\'egende d'une figure ou tableau doit se terminer par un <<.>>!}
\GT{Toute figure doit \^etre r\'ef\'er\'ee dans le texte, donc doit
avoir un label (apr\`es caption) et un ref (dans le texte).}
\GT{Les termes en anglais doivent \^etre mis en \emph{italiques}.}

\gt{Dans mon esprit, un document de <<concept>> et un document de
<<conception>> sont deux choses diff\'erentes.  Les concepts sont plus
abstraits, juste des id\'ees donnant la saveur du jeu.  Alors que la
conception est plus concret --- comment \c{c}a fonctionne. Donc, je
crois qu'ici c'est plus les concepts, donc figure/titre \`a changerù}

\GT{On \'evite <<...>> en fin de phrase. Plut\^ot <<etc.>>.  Et si on
doit utiliser <<...>>, on \'ecrit <<$\backslash$ldots>>.}


\GT{On  \'evite  la  forme  passive  quand  c'est  possible.  Tu  peux
utiliser,  surtout quand  tu d\'ecris  sp\'ecifiquement ce  que  tu as
fait,           le           <<nous          de           modestie>>~:
\url{http://bdl.oqlf.gouv.qc.ca/bdl/gabarit_bdl.asp?id=1706}}



\GT{L'utilisation de $\backslash$gls ne produit pas toujours des
r\'esultats int\'eressants, par ex., ci-bas, avec UML.}

\GT{Et tu ne devrais l'utiliser, quand c'est appropri\'e, que lors de
sa premi\`ere apparition, pas \`a chaque utilisation.}


\GT{On \'evite autant que possible --- sauf exception --- les superlatifs, i.e., <<tr\`es>>!}


\GT{Pour les guillemets en fran\c{c}ais, il faut utiliser les beaux guillemets comme ci-bas.}

\GT{Une citation avec $\backslash$cite doit toujours \^etre
s\'epar\'ee par un espace ins\'ecable (tilde) du texte qui
pr\'ec\`ede.}